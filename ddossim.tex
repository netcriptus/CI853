%!TEX root = ./proposta3.tex

Para a simulação do tráfego necessário, podem ser utilizadas ferramentas que enviam requisições HTTP à aplicação. Como ferramentas, podem ser citados os comandos \emph{curl} e \emph{ab} e a aplicação \emph{LOIC}. A diferença fundamental dentre elas é o nível na qual operam. Enquanto o comando \textbf{curl} opera realizando instâncias singulares de operações simples como \emph{GET}s e \emph{PUT}s, o comando \textbf{ab} automatiza o processo realizando diversas operações de acordo com alguns parâmetros. É possível customizar o nível de concorrência e o intervalo entre as requisições, por exemplo. Tal ferramenta possibilita a medição de algumas métricas como a taxa de entrega de pacotes e o tempo de resposta. Em um nível ainda maior, a aplicação~\cite{loic} foi desenvolvida a fim de realizar ataques de DoS\footnote{Diversos clientes a utilizam a fim de gerar um ataque de DDoS.}. Ela realiza automaticamente a calibragem de diversos parâmetros em níveis inferiores. Entretanto, ela ainda permite a customização de alguns, como a porta de ataque e o protocolo a ser utilizado.

Considerando estas ferramentas, alguns cenários podem ser elaborados para a avaliação. 
%
Para a simulação de clientes legítimos, o comando \emph{curl} pode ser utilizado para criar um \emph{script} que atua como um cliente legítimo. O comando requisitará pela página em questão através de um \emph{GET}. Caso a resposta indique uma mudança de endereço, o \emph{script} será responsável por seguir todas as mudanças e redirecionamentos com chamadas subsequentes do comando \emph{curl}, até que o destino final seja de fato atingido. Deve-se ressaltar que esse comportamento é dificultado em ataques de DDoS, pois eles perderiam muito a sua eficiência ao aguardar por respostas de requisições, para poderem analisar elas, e seguir até o destino final.

Para a obtenção de tempos de resposta e taxa de perda de pacotes da perspectiva do cliente, o comando \emph{ab} pode ser empregado para realizar as medições. Diversas faixas de parâmetros podem ser estipuladas para simular diferentes tipos de comportamentos ou cenários. O uso do \emph{ab} torna-se preferível ao do \emph{curl} quando o objetivo é coletar métricas e não simular um cliente genuíno, afinal, o comando \emph{ab} medirá o desempenho de instâncias \emph{cloud} isoladas, sem realizar qualquer redirecionamento de tráfego. Este aspecto do comportamento do \emph{script} em questão é importante para a diferenciação entre ataques DDoS e \emph{Flash Crowds}~\cite{Thapngam:2011p27061}. Enquanto um ataque é malicioso, uma \emph{Flash Crowd} indica que diversos clientes legítimos estão, de fato, realizando diversas requisições à aplicação. Este caso não será tratado, embora talvez possa ser identificado.

Por fim, para a simulação do ataque DDoS em si, a ferramenta \emph{LOIC} será utilizada, pois ela é uma ferramenta utilizada para ataques reais. Embora a escala em nosso experimento seja muito menor, a operação será baseada em um ataque real e, portanto, condizente com a realidade. Para a obtenção de diferentes endereços IP, diferenciando um DoS de um DDoS, a ferramenta será utilizada simultaneamente por computadores de endereços diferentes.