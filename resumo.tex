%!TEX root = ./proposta3.tex

Ataques de Distributed Denial of Service (DDoS) frequentemente são negligenciados por representarem apenas interrupção de funcionamento normal dos recursos disponíveis. Porém, tratando-se de aplicações destinadas ao comércio eletrônico, uma parada do serviço pode representar grandes perdas financeiras. Com o advento da Internet do Futuro e arquiteturas como a cloud, a mitigação deste tipo de ameaça com o acréscimo de recursos para as aplicações se torna uma alternativa viável, mas que acarreta o problema do e economic DDoS. Este artigo apresenta uma proposta de mecanismo para a mitigação de ataques DDoS direcionados a uma aplicação hospedada em uma cloud. Tal mecanismo é baseado na instanciação de uma réplica da aplicação - operação simples em uma cloud - e redirecionamento de apenas requisições legítimas a esta réplica. O mecanismo é inovador por não precisar identificar os clientes atacantes e, ainda assim, conseguir filtrar apenas o tráfego legítimo sem a carga e possíveis erros de categorização que seriam introduzidos pela tentativa de identificação de clientes.
