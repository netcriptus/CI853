\documentclass[a4paper, 12pt]{article}
\usepackage{sbc-template}

\usepackage{graphicx}
\usepackage[brazil]{babel}
\usepackage[utf8]{inputenc}
\usepackage{epsfig}
\usepackage{float}
\usepackage{graphics}
\usepackage{url}
\usepackage[tight,footnotesize]{subfigure}
\usepackage{stfloats}
\usepackage{enumerate}
% \usepackage[left=3cm,top=3cm,right=3cm,bottom=3cm]{geometry}


% correct bad hyphenation here
\hyphenation{op-tical net-works semi-conduc-tor}


\begin{document}

\title{Arquitetura para Mitigação de Ataques DDoS}

\author{
Cinara Menegazzo\inst{1},
Fernando Cezar Bernardelli\inst{1}, \\
Fernando Henrique Gielow\inst{1},
Nadine Lipa Pari\inst{1}
}
   

   
\address{Departamento de Informática -- Universidade Federal do Paraná\\
  Caixa Postal 19.081 -- 81.531-980 -- Curitiba -- PR -- Brasil
  \email{\{cmenegazzo,fcb06,fhgielow,nelpari\}@inf.ufpr.br}
}     

\maketitle


% {
% \begin{center}
% {\LARGE \textbf{Arquitetura para Mitiga\c{c}\~{a}o de Ataques DDoS}}
% \vskip 0.5cm
% {\Large Cinara Menegazzo, Fernando Cezar Bernardelli, \\ Fernando Henrique Gielow, Nadine
% Lipa Pari}
% \end{center}
% }
\section*{\emph{Resumo}}
\emph{Aqui sera colocado o resumo deste trabalho}
A maioria das propostas para mitigar DDoS são de natureza reativa.

%\section{divisões do artigo}
% 2-trabalhos correlatos - sessão para revisar alguns mecanismos detecção e de mitigação de DDoS
%
%3- CLoud ou entra este tema direto na proxima sessao... 
%4 ou 3 se anterior nao existir - 
%
%Apresentação do servico e modelo de ataque (descrever o contexto da nossa proposta aqui entraria cloud e DDoS em cloud, qual a nosso modelo de ataque, composto por quantos zumbis etc...efeito do ataque, tipo de DDoS que usaremos...) que usaremos para mitigar, aqui entraria blacklist....
%
%Próxima seção descrever o esquema proposto para mitigar DDoS na cloud, nossa arquitetura e como funciona....e tempo requerido para parar um ataque ou detectar um.....aqui se levantaria metricas????
%
%Outra seção para descrever a experimentação(não gosto deste termo) da proposta-talvez análise de desempenho ou e  análise de resultados...modelo de experimentação, etc....
%Conclusao


\section{Introdução}


Diversas pesquisas e propostas têm sido desenvolvidas buscando soluções para problemas da Internet atual, que se propagam para a Internet do Futuro (IF). Tais problemas podem ser amplamente categorizados em nas áreas de mobilidade, qualidade de serviço e segurança, os quais ainda caminham para aceitáveis soluções.

Alguns paradigmas mudaram nestas últimas décadas. Hoje, tem-se os dados e aplicações disponibilizados em localizações físicas distintas e desconhecidas. Outra grande mudança ocorreu na forma de administrar um sistema, que antes era local com seus usuários e servidores característicos. Agora, tal sistema é hospedado em ambientes construídos pelo compartilhamento de recursos de diversos sistemas autônomos (AS) e heterogêneos.

O uso massivo dos recursos disponibilizados na Internet e toda a conectividade que este ambiente computacional proporciona, com serviços para uso pessoal, comercial ou acadêmico, torna este ambiente alvo fácil para códigos maliciosos, agravados pela forma como a arquitetura TCP/IP pode favorecer um atacante. O protocolo IP (\emph{Internet Protocol}) omite as informações da verdadeira identidade de um emissor, essa não autenticação da fonte permite ao atacante realizar o ataque contra a vítima, ainda permanecendo anônimo e impune.

Apesar de esforços de pesquisadores por muitos anos, um tipo de ataque que ainda representa sérias ameaças a muitos servidores na Internet é o ataque de DoS (\emph{Denial of Service}), configurando-se como um dos principais desafios de segurança que atualmente se propaga para a IF. 


A melhor forma de se defender deste tipo de ataque é a prevenção e a reação. O ataque por DoS não visa invadir um computador para obter informações confidenciais, nem tão pouco alterar informações armazenadas nele. Seu objetivo é a indisponibilizar um serviço que está sendo fornecido, utilizando-se do encaminhamento de grandes quantidades de tráfego ao hospedeiro do serviço, de forma que este não consiga responder a todas as requisições que lhe são encaminhadas. 

O problema fica ainda mais crítico quando diversos geradores de tráfego distribuídos intensificam o encaminhamento de tráfego, caracterizando um ataque de DDoS \emph{Distributed Denial of Service} \cite{Sachdeva08ddosincidents}. O resultado obtido é o congelamento, a reinicialização, ou ainda o esgotamento completo de recursos necessários do hospedeiro. Os serviços que mais sofrem com este tipo de ataque são aqueles que permitem requisições anônimas como serviços web. Assim, o desafio de eliminar os ataques de DDoS está na dificuldade de determinar a diferença entre pacotes legítimos e pacotes de atacantes \cite{Li:2009:DDA:1683304.1684620}.


Com as novas arquiteturas de rede e de aplicações que configuram a IF, surgem sistemas complexos e robustos como \emph{clouds}, onde o desafio de mitigar ataques deste tipo torna-se ainda mais desafiador.  O poder de
alocar recursos para suportar um ataque deste tipo em \emph{cloud}  agora torna-se
possível, porém crescem também os custos do usuário para manter tais recursos, o que
carateriza-se como eDDoS (\emph{economic Distributed Denial of Service}) \cite{Soon:10}.
  
A maioria das soluções oferecidas para mitigar DDoS em \emph{cloud} se baseiam inteiramente na maior alocação de
recursos [referenciar???], algumas abordagens diferenciadas se mostram inadequadas por premissas que nem sempre são verdadeiras ou são custosas \cite{Bakshi:10}, \cite{Liu:2010:NFD:1866835.1866849},  colocar mais referencias...... 
Portanto, a mitigação de DDoS em \emph{clouds} ainda demanda pesquisas, comprovado por registros de ataques que abalaram a Internet nos últimos tempos, como os ocorridos contra o Yahoo!, eBay, Amazon.com, e diversos outros \emph{sites} populares em fevereiro de 2000.  No início de 2011 se observou o ataque sofrido pelo hospedeiro de \emph{blogs}, WordPress, que enfrentou o pior ataque de DDoS de sua história \cite{infoexame}. Ataques deste tipo ainda estão por serem disparados e de proporções alarmantes, confirmados por descobertas recentes da engenharia aplicada capaz de gerar redes de zumbis assustadoras, como o caso da rede TDL-4 que é classificada por especialistas em segurança como “não perfeitamente, mas praticamente indestrutível”, com aproximadamente 4,5 milhões de infecções só em 2011 \cite{tdl4}.

O objetivo deste trabalho é elaborar uma arquitetura para mitigação de ataques de
DDoS executados sobre uma aplicação hospedada em uma \emph{cloud}. Esta
arquitetura deverá monitorar o tráfego da aplicação e, quando
detectar ocorrência de ataque de DDoS, será responsável por criar uma nova
instância desta aplicação, garantindo que nenhum tráfego malicioso a alcance. 

Para que este trabalho possa ser.....este artigo está divido conforme 
Na primeira seção estão descritos a motivação e o objetivo deste trabalho, expondo o ambiente de problema e solução que será abordado pela pesquisa. NA seção dois, um levantamento dos trabalhos relacionados é descrito para que se posicione quanto as soluções já existentes. O cenário/arquitetura de .....na seção três

%
%Uma arquitetura de \emph{cloud} pode representar um ambiente propício a ataques por ser usado por diversas pessoas de diversas organizações com suas aplicações e sem o menor controle ou entendimento das configurações ou rede envolvida.
%
%Nos trabalhamos com o targeted attack????
%In a tar-
%geted attack, an adversary wants to gain a critical mass in a
%specific subnet, for example, to attack a specific application
%hosted in that subnet.




\section{Trabalhos Relacionados}

A necessidade de proteger ou mitigar as arquiteturas de rede de ataques de DDoS tem sido reconhecida tanto no meio acadêmico quanto comercial. Segundo \cite{1039856}, três são as linhas de defesa contra ataques de DDoS e compreendem a prevenção, reação ou ambas chamada de defesa híbrida. A prevenção tenta eliminar a possibilidade de ataques de DDoS evitando negação de serviço para os clientes legítimos, enquanto a reação detecta o ataque e responde imediatamente, a abordagem híbrida combina os métodos previnindo e reagindo à ataques.

As pesquisas que envolvem propostas de mitigação para DDoS em arquiteturas de \emph{cloud}, ainda são consideradas incipientes e distantes de uma convergência. Dentre as poucas propostas para estes ambientes destaca-se o \emph{framework} pró-ativo criado por \cite{Hazelhurst:2008:SCU:1456659.1456671}, chamado CluB que considera uma \emph{cloud} como uma rede constituída de um conjunto de \emph{clusters}, ou \emph{Autonomous Systems} (AS) e sugere
que sejam selecionados determinados roteadores para análise de tráfego, dispostos de forma distribuída para evitar que as requisições maliciosas alcancem a aplicação. Estes roteadores ficam responsáveis por gerar \emph{tokens} de autenticaç~ao para legitimar os pacotes e a autenticação corresponde a entrada, saída ou trânsito na arquitetura. Cada \emph{cluster} tem seu código de autenticação e os códigos são trocados periódicamente, podendo ser gerados por função \emph{hash} como MD5 ou SHA. O uso de ferramentas apropriadas de criptografia e periódicas atualizações de componentes da infraestrutura são colocadas como parte da proposta do CluB.



Neste \emph{framework} todo pacote, malicioso ou não, precisa ser verificado para entrar, sair ou transitar na arquitetura, por cada roteador que tiver a responsabilidade de tal tarefa, o que torna-se custoso quanto ao \emph{overhead} causado pela criação da autenticação para cada pacote. Também é necessária a implantação e atualização dos algoritmos de análise de tráfego na arquitetura onde estaria sendo utilizado o CluB, esta questão torna-se inviável ao se tratar de uma arquitetura em \emph{cloud} devido as suas funcionalidades(??? seria a melhor palavra?). 

\cite{Verkaik:2006:PCD:1162666.1162673} apresentam outra proposta pró-ativa, a qual emprega Comunidades de Interesse (COIs) para capturar comportamento coletivo das entidades remotas e as usa para predizer o comportamento futuro. Tal proposta se baseia no fato de que clientes que se tenha tido relacionamentos legítimos anteriormente, possuem bons indícios e podem ser consideradas legítimos. Estas afirmações são geradas da observação de comunicações normais da rede e são utilizadas em conjunto com políticas específicas do cliente para pró-ativamente mitigar os ataques de DDoS usando mecanismos existentes nos roteadores. ALGUEM CONSEGUE VER UM DEFEITO PARA BATER??? 


O trabalho de \cite{Bakshi:10} propõe tratar este cenário de ataques através da criação de uma nova instância da aplicação. Uma vez que um ataque DDoS é detectado, a proposta busca detectar os atacantes através de PINGs - caso um cliente suspeito de ser atacante não responder ao PING, ele é considerado como um atacante, de fato. Desta maneira, apenas os clientes que responderem ao PING serão 
redirecionados para a nova instância da aplicação. Entretanto, tal abordagem depende da premissa que atacantes jamais responderão a PINGs e que clientes genuínos sempre responderão, o que nem sempre condiz com a realidade.

Obviamente, a eficácia de todos os esquemas depende criticamente da capacidade de se identificar os clientes legítimos. 


%Filtro de Bloom

\cite{Walfish:2010:DDO:1731060.1731063} apresenta uma forma de mitigação por ataque classificada como defesa baseada em recursos \cite{Dwork:1992:PVP:646757.705669}. A mitigação emprega o procedimento de que toda vez que um determinado limite de banda seja consumido com requisições para um servidor, este servidor, antes que seus recursos se esgotem, encoraja seus clientes para automaticamente enviar altos volumes de tráfego, como os atacantes já estariam usando a maioria de sua capacidade, eles nao poderiam reagir ao encorajamento. A proposta se baseia na premissa que bons clientes tem condições de aumentar seu consumo de banda e reagir de forma drástica ao encorajamento. O resultado pretendido é que os bons clientes expulsem os maus clientes capturando uma fração maior de recursos do servidor. Um tanto curiosa, esta proposta sugere problemas básicos como a reação apenas quando o servidor já está sendo atacado e o procedimento encoraja que ele receba mais tráfego do que o próprio ataque poderia produzir, ou muitas vezes maior, dificilmente um serviço consiga atender a tantas requisições e quantas requisições inúteis estes clientes deveriam reportar ao servidor, o cliente so seria atendido mediante inúmeras requisições que enviasse ao servidor e não se pode garantir que um atacante não possa, dentro de um determinado periodo de tempo pré-estabelecido, encaminhar diversas requisições todas com o mesmo endereço de origem.


WebSoS \cite{Stavrou:2005:WOS:1090583.1648614} é uma adaptação de \emph{Secure Overlay Services} (SOS) \cite{Keromytis:2002:SSO:964725.633032} que mitiga DDoS em servidores web protegendo-os imediatamenta após a detecção do ataque, com filtragem robusta de tráfego e bloqueio de requisições não aprovadas, formando um \emph{overlay} seguro com o servidor utilizando mecanismos de criptografia de autenticação ou autenticação criptográfica (???? cryptographic authentication mechanismo (Qual seria o nome certo??) ou um teste gráfico de Turing \cite{Dietrich00analyzingdistributed} para diferenciar clientes humanos de \emph{scripts} de ataque. Estes procedimentos, segundo os testes dos autores, não sobrecarregam o funcionamento do serviço, porém exige que sejam configurados os roteadores localizados no perímetro do servidor para controlar o tráfego, procedimento a ser evitado em arquiteturas de \emph{cloud}.
 



\section{Arquitetura para Mitigação de DDoS em \emph{Cloud}}


Uma arquitetura em \emph{cloud} envolve comunicação de inúmeros componentes com APIs (\emph{Application Programing Interface}), geralmente de serviços web. Os usuários desta arquitetura não sabem e não precisam saber sobre a localização de seus dados ou aplicações que desejem utilizar, porém precisam aceitar e dependem  dos níveis de segurança vigentes que é o tópico mais preocupante para administradores.
A segurança em \emph{cloud} compreende dois tipos, segundo \cite{Dhage:2011:IDS:1980022.1980076}:  
\begin{itemize}
\item segurança de dados;
\item segurança da rede.
\end{itemize}

Um ataque em dados afeta um simples usuário enquanto um ataque na rede pode compremeter diversos usuários. Um ataque de DDos em \emph{cloud} compreende um ataque a segurança em rede, portanto é de suma importância. 
Um dos mais importantes fatores para detectar ataques de DDoS é encontrar a correlação entre diferentes fluxos paras um mesmo destino.

Com a possibilidade de novas infraestruturas de recursos como as possiveis pelas \emph{clouds}, cria-se a possibilidade de quando um serviço estiver sendo atacado por DDoS, esta aplicação possa ser movida fisicamente para outro endereço, possibilitando que se tenha uma tolerância a falhas e uma conservação de recursos dispendidos, pois este novo endereçamento só será conhecido por solicitantes legítimos.

Este trabalho propõe uma arquitetura pró-ativa para mitigar este tipo de ataque em \emph{cloud} de forma autônoma e independente. Ele se diferencia de outras propostas pela independência de qualquer aplicação ou protocolo, por prover acurácia na diferenciação entre o tráfego legítimo e o atacante, por não onerar financeiramente o uso dos recursos, e por minimizar os problemas causados pelo ataque sem intervenção humana.

A arquitetura proposta poderá ser utilizada por qualquer aplicação hospedada em uma \emph{cloud}, que ao sofrer indícios de um ataque por DDoS filtra o tráfego ligítimo e encaminha apenas este tr para uma nova instância da mesma aplicação. 

Esta arquitetura, ilustrada na Figura \ref{fig:arq}, é composta por um módulo global chamado de Gerenciador de Tráfego (GF)que não se comunica com a aplicação. Ele filtra o tráfego e envia solicitação de  nova requisição ao cliente para a nova instância da aplicação.(explicar e explicar a filtragem........?????????????? 

\begin{figure}[h!]
\centering
%\includegraphics[width=0.85\textwidth]{arquitetura.eps}
\caption{Ilustração da proposta de arquitetura para mitigação de ataques DDoS}
\label{fig:arq}
\end{figure}

As operações realizadas pelo módulo GF são divididas em quatro sub-módulos:

\begin{enumerate}[i]
  \item Analizador de Tráfego (AT);
  \item Redirecionador de Tráfego(RT);
  \item Gerenciador de \emph{Blacklist}(GB);
  \item Instanciador de Nova Aplicação(INA).
\end{enumerate}

O sub módulo AT observa o comportamento do tráfego de entrada para a aplicação de forma pró-ativa. Focando-se na estimativa de quantidade de tráfego e de processamento no servidor, este sub módulo realiza medição para verificar a existência de um possível ataque de DDoS, se detectado o sub módulo INA é ativado. O INA criará uma nova instância da aplicação em outra instância (ta certo este termo???) de servidor na \emph{cloud}, consequentemente com um endereço IP diferente.
Com isso, o sub módulo RT passará a tratar todo o tráfego de entrada informando ao sub módulo GB, o endereço IP de cada requisição de cliente direcionada ao servidor e encaminhará, a este cliente, uma requisição http (sigla) informando o endereço do novo servidor. 

Tal procedimento é adotado baseando-se no princípio de que o atacante não afetará a nova instância da aplicação, pois ele não interpretará as respostas que informam o IP da nova instância e continuará enviando solicitações em direção ao endereço antigo da aplicação.

 
Por sua vez, o GB nao sei como contar o que ele faz, kkkkkk????? 
 




Focamos na estimativa de quantidade de tráfego para direcionar o tráfego pelo módulo 


Falar sobre:
Banco de dados Redis, chave, valor....
Blacklist com TTL para reenvio de novo endereco da aplicacao, para onde ela foi movida, verificar a questao de performance...
Arquitetura......




% \newpage
% \bibliographystyle{unsrt}
\bibliographystyle{sbc}
\bibliography{proposta3}
\end{document}
