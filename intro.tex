%!TEX root = ./proposta3.tex

Diversas pesquisas e propostas têm sido desenvolvidas buscando soluções para problemas da Internet atual, que se propagam para a Internet do Futuro (IF). Tais problemas podem ser amplamente categorizados nas áreas de mobilidade, qualidade de serviço e segurança, os quais ainda caminham para soluções aceitáveis. % expandir o parágrafo com mais uma frase XXXXXXXXXXXXXXXX

Alguns paradigmas mudaram nestas últimas décadas. Hoje, tem-se os dados e aplicações disponibilizados em localizações físicas distintas e desconhecidas. Outra grande mudança ocorreu na forma de administrar um sistema, que antes era de âmbito mais local, com seus usuários e servidores característicos. Agora, tal sistema é hospedado em ambientes construídos pelo compartilhamento de recursos de diversos sistemas autônomos (AS) e heterogêneos.

O uso massivo dos recursos disponibilizados na Internet e toda a conectividade que este ambiente computacional proporciona com serviços para uso pessoal, comercial ou acadêmico, torna este ambiente um alvo visado para códigos maliciosos. Ademais, isso é agravado pela forma como a arquitetura TCP/IP pode favorecer um atacante. O protocolo IP (\emph{Internet Protocol}) omite as informações da verdadeira identidade de um emissor, essa não autenticação da fonte permite ao atacante realizar o ataque contra a vítima, podendo permanecer anônimo e impune. % UMA REFERENCIA AQUI É IMPORTANTE XXXXXXXXXXXXXXXX

Apesar de muitos anos de esforços de pesquisadores, um tipo de ataque ainda representa sérias ameaças a muitos servidores na Internet: o ataque de  \emph{Denial of Service} (DoS). Ele se configura como um dos principais desafios de segurança atualmente propagado para a IF, que interconecta muito mais dispositivos e indivíduos. 


Uma forma de se defender deste tipo de ataque é a prevenção e a reação. O ataque por DoS não visa invadir um computador para obter informações confidenciais, nem tão pouco alterar informações armazenadas nele. Seu objetivo é a indisponibilização de um serviço que está sendo fornecido, utilizando-se do encaminhamento de grandes quantidades de tráfego ao hospedeiro do serviço. Desta forma, este serviço não conseguirá responder a todas as requisições que lhe são encaminhadas. 

O problema se torna ainda mais severo quando diversos geradores de tráfego intensificam o encaminhamento de tráfego de maneira distribuída, caracterizando um ataque de \emph{Distributed Denial of Service} (DDoS \cite{Sachdeva08ddosincidents}). O resultado obtido é o congelamento, a reinicialização, ou ainda o esgotamento completo de recursos necessários ao hospedeiro. Os serviços que mais sofrem com este tipo de ataque são aqueles que permitem requisições anônimas, como serviços web. Assim, o desafio de eliminar os ataques de DDoS está na dificuldade de determinar a diferença entre pacotes legítimos e pacotes de atacantes \cite{Li:2009:DDA:1683304.1684620}.


Com as novas arquiteturas de rede e de aplicações que configuram a IF, surgem sistemas complexos e robustos como \emph{clouds}, onde o desafio de mitigar ataques deste tipo torna-se ainda mais necessário.  Embora o poder de
alocar recursos para suportar um ataque deste tipo  agora torna-se
possível em um \emph{cloud}, crescem com isso também os custos do usuário para manter tais recursos. Tal reação
carateriza o chamado \emph{economic DDoS} (eDDoS \cite{Soon:10}).
  
A maioria das soluções comumente oferecidas para mitigar DDoS em \emph{cloud} se baseiam inteiramente na maior alocação de
recursos \cite{Peng:2007:SND:1216370.1216373}.  %% XXX
%
Algumas abordagens diferenciadas se mostram inadequadas por premissas que nem sempre são verdadeiras ou por serem custosas demais \cite{Bakshi:10}, \cite{Liu:2010:NFD:1866835.1866849}.
%,  colocar mais referencias...... 
Existem diversos registros de ataques que abalaram a Internet nos últimos tempos, como os ocorridos contra o Yahoo!, eBay, Amazon.com e diversos outros \emph{sites} populares em fevereiro de 2000.  No início de 2011 se observou o ataque sofrido pelo hospedeiro de \emph{blogs}, WordPress, que enfrentou o pior ataque de DDoS de sua história \cite{infoexame}. Ataques deste tipo ainda são disparados em proporções alarmantes, de acordo com descobertas recentes que também revelam a engenharia aplicada que gera redes de zumbis. Um exemplo destas é a rede TDL-4, que é classificada por especialistas em segurança como “não perfeitamente, mas praticamente indestrutível”, com aproximadamente 4,5 milhões de infecções só em 2011 \cite{tdl4}. Assim, a mitigação de DDoS em \emph{clouds} ainda demanda pesquisas.

O objetivo deste trabalho é elaborar uma arquitetura para mitigação de ataques de
DDoS executados contra uma aplicação hospedada em uma \emph{cloud}. Esta
arquitetura deverá monitorar o tráfego da aplicação e, quando
detectar a possível ocorrência de um ataque de DDoS, criará uma nova
instância desta aplicação, garantindo que nenhum tráfego malicioso a alcance. 

\textbf{CHANGEME},
Este trabalho está divido conforme 
Na primeira seção estão descritos a motivação e o objetivo deste trabalho, expondo o ambiente de problema e solução que será abordado pela pesquisa. NA seção dois, um levantamento dos trabalhos relacionados é descrito para que se posicione quanto as soluções já existentes. O cenário/arquitetura de .....na seção três

%
%Uma arquitetura de \emph{cloud} pode representar um ambiente propício a ataques por ser usado por diversas pessoas de diversas organizações com suas aplicações e sem o menor controle ou entendimento das configurações ou rede envolvida.
%
%Nos trabalhamos com o targeted attack????
%In a tar-
%geted attack, an adversary wants to gain a critical mass in a
%specific subnet, for example, to attack a specific application
%hosted in that subnet.
