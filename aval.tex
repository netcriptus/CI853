%!TEX root = ./proposta3.tex


%A avaliação de um sistema de defesa para ataques pode ser avaliado sobre os aspectos de sua construção. 
% WTF esse trecho acima? xD
%
Uma solução para mitigar ataques pode ser avaliada quanto a sua capacidade de detectar ataque, assim como, a de reagir a ele. Outra abordagem para avaliar uma solução é quanto a sua capacidade de manter as condições normais de funcionamento do cenário de ataque, mesmo sob ataque. Segundo \cite{4600003} é importante para um sistema de defesa estimar diversos aspectos como: custo de desenvolvimento, desempenho, degradação do serviço e custo de robustez. 

A maioria das métricas para calcular o impacto de ataques DDoS estão relacionadas com as medidas de eficiência dos padrões de defensa \cite{4809152}. Atualmente, são consideradas estratégias de medição da quantidade de tráfego legítimo passando \textbf{pela canal com e sem ataque (NAO ENTENDI)}. Outros trabalhos tem se concentrado na medida do tempo de resposta para avaliar a eficiência de uma solução. \cite{Mirkovic:2007:TUM:1281700.1281708} utiliza um modelo baseado em \emph{threshold} como métrica para aferir o impacto de DDoS. Quando uma medida excede este \emph{threshold}, ocorre a  indicação da baixa qualidade do serviço. Esta medida é indicada para aplicações fim-a-fim como http.


A etapa de avaliação do nosso mecanismo consistirá principalmente na análise
da capacidade do servidor em atender novas requisições. Se o ataque de DDoS for devidamente
mitigado, as requisições de atacantes serão ignoradas, após a inclusão do requisitante na \emph{blacklist}. Assim, o servidor na \emph{cloud} deverá ser capaz de redirecionar apenas clientes legítimos 
para a nova instância e garantirá que eles terão acesso direto nas próximas requisições. %ISTO CABE A PARTE DE ARQUITETURA.....Com isso, para a análise, serão utilizadas ferramentas que geram requisições ao servidor. Como exemplos, o comando 
%\emph{curl} pode ser utilizado em linha de comando para gerar uma requisição HTTP, sendo necessária a criação de \emph{scripts} mais robustos para incorporar diversas requisições. Outra alternativa é o uso do comando \emph{ab}, que permite a especificação do número de requisições que devem ser realizadas, a concorrência destas requisições, o tempo máximo que se deve esperar por respostas, e diversos outros parâmetros.
Portanto, de acordo com as métricas especificadas por \cite{4600003}, este trabalho de pesquisa fará uso de métricas como o tempo de resposta do servidor para requisições atendidas, a taxa de requisições atendidas com relação ao número de clientes, a carga gerada pelos módulos da arquitetura de acordo com o número de clientes \footnote{o termo ``clientes'', usado neste parágrafo engloba tanto clientes legítimos quanto atacantes.}. 
%Cabe ressaltar que, este trabalho é desenvolvido no sentido de que, sempre que um grande tráfego de pacotes seja destinado à aplicação, a arquitetura de defesa, mediante avaliação de comprometimento de recursos disponíveis na mesma, trabalhará para criar uma nova instância, independente de se ter uma acurácia  na caracterização de ataque de DDoS. Portanto, a avaliação quanto a geração de falsos positivos não se fará necessária.
Cabe ressaltar que este trabalho não necessita de uma previsão muito grande na detecção no sentido de que é melhor realizar uma calibragem muito sensível e possuir falsos positivos do que possuir falsos negativos. Isto ocorre devido à natureza do nosso mecanismo. Se uma nova instância for criada e o trafego for redirecionado à toa, o custo serão alguns mínimos milisegundos de latência. Caso o mecanismo não detecte um ataque, o custo será muito mais significativo. Desta maneira, a avaliação quanto à falsos positivos não será tão necessária. % quanto a avaliação não detecção do ataque. 
