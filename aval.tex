A etapa de avaliação da proposta consistirá principalmente na análise
da capacidade do servidor de atender novas requisições. Se o ataque de DDoS for devidamente
mitigado, as requisições de atacantes serão ignoradas, após eles serem introduzidos na \emph{blacklist}
e, assim, o servidor na \emph{cloud} deverá ser capaz de redirecionar apenas clientes legítimos 
para o novo servidor e garantirá acesso direto à ela.

Com isso, para a análise, serão utilizadas ferramentas que geram requisições ao servidor. Como exemplos, o comando 
\emph{curl} pode ser utilizado em linha de comando para gerar uma requisição HTTP, sendo necessária a criação de \emph{scripts} mais robustos para incorporar diversas requisições. Outra alternativa é o uso do comando \emph{ab}, que permite a especificação do número de requisições que devem ser realizadas, a concorrência destas requisições, o tempo máximo que se deve esperar por respostas, e diversos outros parâmetros.

Como métricas, poderão ser utilizados o \textbf{tempo de resposta do servidor para requisições atendidas}, a \textbf{taxa de requisições atendidas com relação ao número de clientes}, a \textbf{carga gerada pelos módulos da arquitetura de acordo com o número de clientes}\footnote{note que o termo ``clientes'', aqui, engloba tanto clientes legítimos quanto atacantes.}. Novas métricas podem vir a serem consideradas, de acordo com a pesquisa que se sucederá.
