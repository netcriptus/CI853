%!TEX root = ./proposta3.tex

A avaliação de um sistema de defesa para ataques pode ser avaliado sobre os aspectos de sua construção. Uma solução para mitigar ataques pode ser medido quanto a sua capacidade de detectar o tipo do ataque, assim como, a de reagir a eles. Outra abordagem que avalia uma solução é quanto a sua capacidade em manter as condições normais de funcionamento do cenário de ataque, mesmo sob ataque. Segundo \cite{4600003} é importante estimar um sistema de defesa sobre diversos aspectos como: custo de desenvolvimento, desempenho, degradação do serviço e custo de robustez. 

A maioria das métricas para calcular o impacto de ataques DDoS estão efetivamente relacionadas com as medidas de eficiência dos padrões de defensa \cite{4809152}. Atualmente, são consideradas estratégias de mediç~ao da quantidade de tráfego legítimo passando pela canal com e sem ataque. Outros trabalhos tem se concentrado na medida do tempo de resposta para avaliar a eficiencia de uma solução. cite{Mirkovic:2007:TUM:1281700.1281708} utiliza como métrica para aferir o impacto de DDoS, um modelo baseado em \emph{threshold}. Quando uma medida excede este \emph{threshold}, ocorre a  indicação da baixa qualidade do serviço, esta medida é indicada para aplicações fim-a-fim como http.
