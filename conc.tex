Este trabalho apresentou uma proposta de arquitetura de mitigação para ataques de DDoS direcionados à aplicações web hospedadas em \emph{cloud}. A arquitetura é dependente apenas da existência do ambiente na \emph{cloud}. Os quatro sub módulos especificados garantem o funcionamento das etapas de forma autônoma até mesmo para gerar uma nova instância da aplicação. Assim como, permitirá livre acesso aos clientes legítimos, desde que atendam um redirecionamento solicitado. O uso da \emph{blacklist} terá gerencimento eficiente pelo uso de tempo de validade, que no caso de atacantes, terá aumento exponencial para reincidências.

