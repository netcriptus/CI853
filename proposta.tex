\documentclass[a4paper, 12pt]{article}
\usepackage{templates/sbc-template}
\usepackage{color}
\usepackage{graphicx}
\usepackage[brazil]{babel}
\usepackage[utf8]{inputenc}
\usepackage{epsfig}
\usepackage{float}
\usepackage{graphics}
\usepackage{url}
\usepackage[tight,footnotesize]{subfigure}
\usepackage{stfloats}
\usepackage{enumerate}
% \usepackage[left=3cm,top=3cm,right=3cm,bottom=3cm]{geometry}


% correct bad hyphenation here
\hyphenation{op-tical net-works semi-conduc-tor}


\begin{document}

\title{Arquitetura para Mitigação de Ataques DDoS}

\author{
Cinara Menegazzo\inst{1},
Fernando Cezar Bernardelli\inst{1}, \\
Fernando Henrique Gielow\inst{1},
Nadine Lipa Pari\inst{1}
}
   

   
\address{Departamento de Informática -- Universidade Federal do Paraná\\
NR2 - Núcleo de Redes Sem Fio e Redes Avançadas\\
  Caixa Postal 19.081 -- 81.531-980 -- Curitiba -- PR -- Brasil
  \email{\{cmenegazzo,fcb06,fhgielow,nelpari\}@inf.ufpr.br}
}     

\maketitle


\begin{abstract}
%!TEX root = ../proposta.tex

Ataques de Distributed Denial of Service (DDoS) frequentemente são negligenciados por representarem apenas interrupção de funcionamento normal dos recursos disponíveis. Porém, tratando-se de aplicações destinadas ao comércio eletrônico, uma parada do serviço pode representar grandes perdas financeiras. Com o advento da Internet do Futuro e arquiteturas como a cloud, a mitigação deste tipo de ameaça com o acréscimo de recursos para as aplicações se torna uma alternativa viável, mas que acarreta o problema do economic DDoS. Este artigo apresenta uma proposta de mecanismo para a mitigação de ataques DDoS direcionados a uma aplicação hospedada em uma cloud. Tal mecanismo é baseado na instanciação de uma réplica da aplicação - operação simples em uma cloud - e redirecionamento de apenas requisições legítimas a esta réplica. O mecanismo é inovador por não precisar identificar os clientes atacantes e, ainda assim, conseguir filtrar apenas o tráfego legítimo sem a carga e possíveis erros de categorização que seriam introduzidos pela tentativa de identificação de clientes.

\end{abstract}



\section{Introduction}
%!TEX root = ./proposta3.tex

Diversas pesquisas têm sido desenvolvidas para tratar questões da Internet atual, que podem se propagar para a Internet do Futuro (IF). Tais problemas podem ser amplamente categorizados nas áreas de mobilidade, qualidade de serviço e segurança, os quais ainda caminham para soluções aceitáveis, agravados pelo surgimento de novas arquiteturas. Hoje tanto os dados quanto as aplicações são oferecidos em localizações físicas distintas e desconhecidas. Outra grande mudança ocorreu na forma de administrar um sistema, que antes era de âmbito mais local, com seus usuários e servidores característicos, e agora esses sistemas são hospedado em ambientes construídos pelo compartilhamento de recursos de diversos sistemas autônomos (AS) e heterogêneos~\cite{5486552}.

%O uso massivo dos recursos disponibilizados na Internet e toda a conectividade que este ambiente computacional proporciona com serviços para uso pessoal, comercial ou acadêmico, torna este ambiente um alvo visado para códigos maliciosos. Ademais, isso é agravado pela forma como a arquitetura TCP/IP pode favorecer um atacante. O protocolo IP (\emph{Internet Protocol}) omite as informações da verdadeira identidade de um emissor, essa não autenticação da fonte permite ao atacante realizar o ataque contra a vítima, podendo permanecer anônimo e impune~\cite{1039856}.

Apesar de muitos anos de esforços em pesquisas, os ataques de \emph{Denial of Service} (DoS) ainda representam sérias ameaças a muitos servidores na Internet.  Ele se configura como um dos principais desafios de segurança atualmente propagado para a IF, que interconectará muito mais dispositivos e indivíduos.  %Uma forma de se defender de ataques DoS é a prevenção e a reação. 
O ataque por DoS não visa invadir um computador para obter informações confidenciais, nem tão pouco alterar informações armazenadas nele. Seu objetivo é a indisponibilização de um serviço fornecido, utilizando-se do encaminhamento de grandes quantidades de tráfego ao hospedeiro do serviço. %Desta forma, este serviço não responderá a todas as requisições que lhe são encaminhadas. 
Essa questão torna-se ainda mais severa quando diversos geradores de tráfego intensificam o encaminhamento de tráfego de maneira distribuída, caracterizando um ataque de \emph{Distributed Denial of Service} (DDoS) \cite{Sachdeva08ddosincidents}. 
%
Embora tal carga seja um problema apenas momentâneo, em se tratando de aplicações destinadas ao comércio eletrônico, por exemplo, uma parada do serviço representa grandes perdas financeiras. 
%A consequência é o congelamento, a reinicialização, ou ainda o esgotamento completo de recursos necessários ao hospedeiro. Os serviços que mais sofrem com este tipo de ataque são aqueles que permitem requisições anônimas, como serviços web. Assim, o desafio de eliminar os ataques de DDoS está na dificuldade de determinar a diferença entre pacotes legítimos e pacotes de atacantes \cite{Li:2009:DDA:1683304.1684620}.

Com as novas arquiteturas de rede e de aplicações que configuram a Internet, têm surgido sistemas complexos e robustos como \emph{Clouds} (Nuvens), onde o desafio de mitigar ataques DoS torna-se ainda mais necessário. A maioria das soluções comumente oferecidas para mitigar DDoS em \emph{cloud} se baseia na maior alocação de recursos \cite{Peng:2007:SND:1216370.1216373}, porém elas também aumentam os custos do usuário para manter tais recursos. Este comportamento carateriza \emph{economic DDoS} (eDDoS) \cite{Soon:10}.  
%
Assim, essas abordagens tornam-se inadequadas pois a premissa de alocação de recursos nem sempre é viável por ser custosa \cite{Bakshi:10}, \cite{Liu:2010:NFD:1866835.1866849}. %devido a premissas que nem sempre viáveis %como supor que um cliente legítimo sempre responderá a PINGs, ou 
% por serem custosas demais \cite{Bakshi:10}, \cite{Liu:2010:NFD:1866835.1866849}. 
%Existem diversos registros de ataques que abalaram a Internet nos últimos tempos, como os ocorridos contra o Yahoo!, eBay, Amazon.com e diversos outros \emph{sites} populares em fevereiro de 2000.  
%No início de 2011 se observou o ataque sofrido pelo hospedeiro de \emph{blogs}, WordPress, que enfrentou o pior ataque de DDoS de sua história \cite{infoexame}.
% Ataques de DDoS são disparados em proporções alarmantes, de acordo com descobertas recentes que também revelam a engenharia aplicada que gera redes de zumbis. %Um exemplo destas é a rede TDL-4, que é classificada por especialistas em segurança como “não perfeitamente, mas praticamente indestrutível”, com aproximadamente 4,5 milhões de infecções só em 2011 \cite{tdl4}.
% A partir de fevereiro 2010, o grupo ativista \textit{hacker} conhecido como \textit{Anonymous} começou uma série de ataques de cunho político e ideológico contra várias instituições de porte internacional \cite{titstorm}. A parte mais massiva desses ataques era constituída de DDoS.
% \cite{infoexame}.  Assim, a mitigação de DDoS em \emph{clouds} ainda demanda pesquisas.

%A necessidade de proteger ou mitigar as arquiteturas de rede de ataques de DDoS tem sido reconhecida tanto no meio acadêmico quanto comercial. 
%Segundo \cite{1039856}, três seriam as linhas clássicas de defesa contra ataques de DDoS, compreendendo a \textbf{prevenção}, a \textbf{reação}, ou ambas, na chamada defesa \textbf{híbrida}. A prevenção tenta eliminar a possibilidade de ataques de DDoS, isto é, evita a negação de serviço para os clientes legítimos. A linha de reação detecta o ataque e responde imediatamente, e a abordagem híbrida combina os métodos anteriores, não só prevenindo mas também reagindo à ataques. 
%A maioria das soluções de segurança são construídas focadas na prevenção de ataques \cite{4429182}. 
%As abordagens clássicas buscam reduzir os riscos a zero, o que é impraticável para todos os tipos de riscos, pois podem ser custosos e complexos.
%Contudo, os atuais sistemas em redes são tão complexos que é impossível identificar e corrigir todas as suas vulnerabilidades antes que elas se tornem ataques ou intrusões e, muitas vezes, impossível de se recuperar de falhas decorrentes. Logo, a garantia de funcionamento destes sistemas sob condições de ataques ou intrusões têm se tornado uma prioridade. Diversos autores, como \cite{Verissimo} e \cite{4796927}, têm defendido que as linhas clássicas de defesa não apresentam abordagens que garantam a tolerância a falhas em caso de ataques ou intrusões. Assim, a abordagem de \textbf{tolerância a intrusão}, a qual lida com sistemas de segurança que garantem o funcionamento dos sistemas mesmo que estejam sob problemas de segurança, minimizando ao máximo os prejuízos até o retorno ao fluxo normal de funcionamento \cite{Fraga_Powell_1985}. Ao invés de tentar se prevenir de todo tipo de ataque ou intrusão, os sistemas passam a tolerar os problemas simples de segurança e criam mecanismos de controle à intrusão  de forma que eles não causem nenhuma falha ao sistema. Esta abordagem faz uso de replicação ou redundância para garantir os aspectos de tolerância e são aplicados,especificamente, quando a forma de intrusão ou ataque seja diferenciada ou desconhecida \cite{4796927}. 

Este trabalho propõe uma arquitetura reativa e tolerante a falhas para a mitigação de ataques de DDoS executados contra aplicações hospedadas em uma \emph{cloud}. Tal arquitetura é baseada na instanciação de uma réplica da aplicação - operação simples em uma cloud - e redirecionamento de apenas requisições legítimas a esta réplica.  A arquitetura monitora o tráfego de uma aplicação e ao detectar uma possível anomalia, isto é, a ocorrência de um ataque de DDoS, ela estabelece uma nova instância desta aplicação, garantindo que nenhum tráfego malicioso a alcance. 
As diferenças desta solução para outras propostas são que a aplicação hospedada não precisa prover acurácia na filtragem de tráfego legítimo, o uso dos recursos não é onerado financeiramente, e a intervenção humana é desnecessária. 
%A solução é inovadora porque não precisa identificar os clientes atacantes e, ainda assim, consegue filtrar apenas o tráfego legítimo sem a carga e possíveis erros de categorização que seriam introduzidos pela tentativa de identificação de clientes. 
Uma avaliação experimental considerando o tempo de resposta aos clientes %, bem como a sobrecarga ao sistema, 
mostra a eficácia do implementação da arquitetura diante de ataques DDoS a um serviço Web.
 
%Embora os ataques de DDoS ainda persistam e tenham crescido ultimamente, dadas as capacidades das redes, este tipo de ataque pode ser considerado comum e bem conhecido pelos sistemas de segurança existentes executando em arquiteturas comuns. Por este motivo, o seu tratamento pode ser incluído dentro da abordagem clássica. Assim, este trabalho de pesquisa propõe uma arquititetura de mitigação classificada como reativa, pois verifica o tráfego e reage as suas anomalias, e tolerante a falhas por garantir o funcionamento da aplicação mesmo sob ataque, através de replicação da aplicação original que passa a servir apenas como redirecionador de tráfego.

O restante do artigo está organizado da seguinte maneira: a Seção 2 apresenta os trabalhos relacionados. A Seção 3 detalha a arquitetura proposta para a mitigação de ataques \emph{DDoS}. A Seção 4 apresenta uma descrição da implementação realizada da arquitetura. A Seção 5 apresenta uma avaliação, juntamente com o cenário e os resultados. Por fim, a conclusão e trabalhos futuros são apresentados na Seção 6.


\section{Related Works}
%!TEX root = ./proposta.tex


As pesquisas que envolvem propostas de mitigação de DDoS em arquiteturas de \emph{cloud}, ainda são consideradas incipientes e distantes de uma convergência. Dentre as poucas propostas para estes ambientes, destaca-se o \emph{framework} pró-ativo CluB, apresentado em \cite{Hazelhurst:2008:SCU:1456659.1456671}, que considera uma \emph{cloud} como uma rede constituída de um conjunto de \emph{clusters} ou AS. Este trabalho sugere
que sejam selecionados determinados roteadores, dispostos de forma distribuída, para análise de tráfego e consequente prevenção de que as requisições maliciosas alcancem a aplicação. Estes roteadores são responsáveis por gerar \emph{tokens} de autenticação para legitimar os pacotes, sendo que a autenticação é necessária para a entrada, saída ou trânsito na arquitetura. Cada \emph{cluster} tem seu código de autenticação, que é trocado periodicamente, podendo ser gerado por uma função \emph{hash}, como MD5 ou SHA. O uso de ferramentas apropriadas de criptografia e atualizações periódicas de componentes da infraestrutura fazem parte da proposta do CluB. Neste \emph{framework}, todo pacote, malicioso ou não, precisa ser verificado para entrar, sair ou transitar na arquitetura. Cada roteador alocado  deverá realizar a verificação, o que é custoso devido ao \emph{overhead} causado pela autenticação de cada pacote. Também é necessária a implantação e atualização dos algoritmos de análise de tráfego na arquitetura onde estaria sendo utilizado o CluB. Esta questão se torna inviável ao se tratar de uma \emph{cloud}, devido à nebulosidade de sua arquitetura e infraestrutura.

\cite{Verkaik:2006:PCD:1162666.1162673} apresentam outra proposta pró-ativa, que emprega Comunidades de Interesse (COIs) para capturar dados sobre o comportamento coletivo das entidades remotas, utilizando-os para predizer o comportamento futuro. Tal proposta se baseia no fato de que clientes que tiveram relacionamentos legítimos anteriormente possuem bons indícios e podem ser considerados novamente legítimos. Estas afirmações são geradas da observação de comunicações normais da rede e são utilizadas em conjunto com políticas específicas do servidor para mitigar pró-ativamente os ataques de DDoS, usando mecanismos existentes nos roteadores. 
%
Entretanto, a identificação dos clientes passados não é tão trivial. Além do pequeno \emph{overhead} gerado pela verificação, endereços IPs são normalmente dinâmicos e a exigência da realização de \emph{login} para a identificação não é possível, dado que o ataque de DDoS pode a impossibilitar.


Em \cite{Bakshi:10}, os ataques são tratados através da criação de uma nova instância da aplicação. Uma vez que um ataque DDoS é detectado, a proposta busca identificar os atacantes através de PINGs: caso um cliente suspeito de ser atacante não responda ao PING, ele é considerado como um atacante, de fato. Desta maneira, apenas os clientes que responderem ao PING serão 
redirecionados para a nova instância da aplicação. Entretanto, tal abordagem depende da premissa que atacantes jamais responderão a PINGs e que clientes genuínos sempre responderão, o que nem sempre condiz com a realidade.



\cite{Walfish:2010:DDO:1731060.1731063} apresenta uma forma de mitigação de ataque classificada como defesa baseada em recursos \cite{Dwork:1992:PVP:646757.705669}. %A mitigação emprega o procedimento de que 
Toda vez que um determinado limite de banda é consumido com requisições para um servidor, este servidor, antes que seus recursos se esgotem, encoraja seus clientes a enviar volumes ainda mais altos de tráfego. Considera-se que os atacantes já estariam usando sua capacidade máxima e, assim, eles não poderiam reagir ao encorajamento. A proposta se baseia na premissa que bons clientes têm condições de aumentar seu uso de banda e reagir de forma drástica ao encorajamento. O resultado pretendido é que os bons clientes dominem os maus clientes ao capturar uma fração maior de recursos do servidor. O cliente será atendido caso ele tenha banda o suficiente para se sobressair mediante o tráfego dos atacantes. Um tanto curiosa, esta proposta ocasiona diversos problemas como %a reação apenas quando o servidor já está sendo atacado e o procedimento 
o encorajamento a recebimento de ainda mais tráfego em cenários de ataque. 
% do que o próprio ataque poderia produzir, ou muitas vezes maior, 
Difícilmente um serviço conseguirá atender a tantas requisições e clientes legítimos não necessariamente dominarão o tráfego que chega ao servidor.


Obviamente, a eficácia de todos os esquemas depende criticamente da capacidade de se identificar ou filtrar os clientes legítimos. 
%
WebSoS \cite{Stavrou:2005:WOS:1090583.1648614} é uma adaptação de \emph{Secure Overlay Services} (SOS) \cite{Keromytis:2002:SSO:964725.633032} que mitiga DDoS em servidores web, reativamente após a detecção do ataque. Com uma filtragem robusta de tráfego e bloqueio de requisições~não aprovadas, forma-se um \emph{overlay} seguro. O servidor utiliza mecanismos de autenticação criptográfica e um teste gráfico de Turing \cite{Dietrich00analyzingdistributed} para diferenciar clientes humanos de \emph{scripts} de ataque. Estes procedimentos, segundo os testes dos autores, não sobrecarregam o funcionamento do serviço, porém exigem que os roteadores localizados no perímetro do servidor sejam configurados para controlar o tráfego, procedimento inviável para arquiteturas de \emph{cloud}.


\section{Architecture for DDoS Mitigation in Clouds}
%!TEX root = ./proposta3.tex


%Uma estrutura de \emph{cloud} envolve a comunicação entre inúmeros componentes de serviços web. Os usuários desta arquitetura não sabem e não precisam saber sobre a localização de seus dados ou as aplicações que desejem utilizar, porém precisam aceitar e dependem  dos níveis de segurança vigentes, que são tópicos preocupantes para os administradores. %A segurança em \emph{cloud} compreende as áreas de segurança de dados e da rede, segundo \cite{Dhage:2011:IDS:1980022.1980076}. 

% Enquanto um ataque em dados afeta um número restrito de usuários, um ataque na rede pode comprometer diversos usuários simultaneamente. Como ataques de DDos em uma \emph{cloud} compreendem um ataque à segurança em rede, eles são de importância crítica. Com a nova infraestrutura de recursos criada pelas \emph{clouds}, existe a possibilidade de se mover fisicamente uma aplicação para outro endereço quando ela é atacada por DDoS. Com isso, pode-se garantir tolerância à falhas e conservação de recursos despendidos, pois este novo endereço só será conhecido por solicitantes legítimos.

Este trabalho propõe uma arquitetura para mitigar ataques de DDoS em \emph{clouds} de forma autônoma e independente. A arquitetura proposta pode ser utilizada por qualquer aplicação web hospedada em uma \emph{cloud} que, ao sofrer indícios de um ataque DDoS, filtra o tráfego legítimo e encaminha apenas este para uma nova instância da mesma aplicação. 

Esta arquitetura, ilustrada na Figura \ref{fig:arq}, é composta por um módulo geral chamado de Gerenciador de Tráfego (GT), que não se comunica diretamente com a aplicação. Esse módulo possui os submódulos INA, GB, AT e RT. Além disso, a instância do banco de dados (BD) é exterior às demais instâncias da \emph{cloud}, visto que o banco de dados também está nas nuvens e pode ser acessado de qualquer outra instância \emph{cloud}. 
% GF virou GT XXXXX

\begin{figure}[t!]
  \centering
  \begin{minipage}[b]{0.59\linewidth}
    \centering
    \includegraphics[width=0.8\textwidth]{images/arq.eps}
	\caption{Arquitetura de mitigação de DDoS}
	\label{fig:arq}
  \end{minipage}
  \begin{minipage}[b]{0.4\linewidth}
    \centering
    \includegraphics[width=0.85\textwidth]{images/ddos-dir.pdf}
	\caption{Fluxo de tráfego}
	\label{fig:traf-ddos}
  \end{minipage}

\end{figure}


% \begin{figure}[h!]
% \centering
% \includegraphics[width=0.45\textwidth]{images/arq.eps}
% \caption{Arquitetura para mitigação de ataques DDoS}
% \label{fig:arq}
% \end{figure}

O submódulo AT observa o comportamento do tráfego de entrada para a aplicação de forma pró-ativa. Ele foca na estimativa de quantidade de tráfego e de processamento no servidor, e realiza medição para detectar a existência de um possível ataque DDoS. Caso um ataque seja detectado, o submódulo INA é ativado. O INA criará uma nova instância da aplicação em outro servidor na \emph{cloud}, consequentemente com um endereço IP diferente. %\footnote{Com a criação da nova instância da aplicação, a antiga é desativada. A primeira instância da \emph{cloud} servirá apenas para redirecionar o tráfego}.
O submódulo RT trata todo o tráfego de entrada, respondendo com um redirecionamento para a nova instância da aplicação, como visto na Figura~\ref{fig:traf-ddos}. O RT parte do princípio que os atacantes DDoS não interpretam as respostas obtidas do servidor, pois se interpretarem, sua eficiência é reduzida. Desta maneira, apenas os clientes legítimos (Nós \emph{C} na figura) serão, de fato, redirecionados à nova aplicação, enquanto os nós atacantes não passarão pelo RT.

Ao tentar redirecionar os clientes para a nova instância, o endereço do cliente, seja ele legítimo ou não, será adicionado em uma \emph{blacklist}. Os clientes presentes nesta lista têm suas requisições descartadas, a fim de reduzir o custo de processamento de respostas no servidor. Entretanto, como o cliente legítimo é informado do redirecionamento antes de seu endereço entrar na \emph{blacklist}, ele terá acesso à esta nova instância e poderá enviar uma nova requisição. Registros com tempo de validade são empregadas nesta \emph{blacklist}, dado que respostas podem ser perdidas. O tempo de validade na lista aumenta exponencialmente, para diminuir ainda mais a sobrecarga. Cabe ao GB, o papel de adicionar e gerenciar a saída de endereços de clientes à \emph{blacklist}, assim como o tempo de validade da entrada que aumenta exponencialmente.

% Contudo, para prevenir que este controle impeça o acesso de clientes legítimos nas próximas requisições, o cliente, ao ser direcionado para a nova instância, terá este endereço armazenado na forma de \emph{cookies} em sua máquina. Este procedimento garante que apenas clientes legítimos tenham conhecimento do novo endereço da aplicação, dado que atacantes de DDoS não irão manter \emph{cookies}. Por fim,  tal processo de reinstanciação de aplicação e redirecionamento de tráfego pode ser repetido recursivamente, até um dado número máximo de redirecionamentos.

% A Figura~\ref{fig:cen} ilustra um cenário sob ataque de DDoS, sendo que os clientes são representados pelos ícones dos diversos navegadores, e a nave é o logo da aplicação LOIC (\emph{Low Orbit Ion Cannon}). À esquerda, todos eles enviam seu tráfego para o que imaginam ser a instância da aplicação. Considerando um cenário sob ataque, a aplicação será replicada, e a sua instância original servirá para redirecionar o tráfego legítimo até a aplicação nova. À direita, é ilustrado o resultado do redirecionamento: clientes genuínos conseguem atingir a instância nova da aplicação, enquanto os atacantes mantém o ataque na antiga instância de \emph{cloud}, que agora opera apenas redirecionando tráfego.
% 
% 
% \begin{figure}[t!]
% 	\centering
% 	\includegraphics[width=0.40\textwidth]{images/an1.eps}
% 	% \caption{bla}
% 	\hskip 1cm
% 	\includegraphics[width=0.40\textwidth]{images/an2.eps}
% 	\caption{Comportamento do tráfego em um cenário sob ataque}
% 	\label{fig:cen}
% \end{figure}


\section{Implementation}
Para a implementação da arquitetura, a solução em \emph{cloud}~\cite{heroku} foi utilizada. Ela oferece infra-estrutura como serviço de hospedagem, possibilitando o desenvolvimento em \emph{Ruby on Rails}. %Dentre estas alternativas, a implementação foi realizada em \emph{Ruby on Rails} (RoR), por maior experiência de membros do grupo e pela versatilidade da linguagem, que se mostra mais direta para a implementação, embora qualquer outro \emph{framework} e linguagem pudessem ser utilizados.
%
%
A arquitetura do \emph{framework}~\cite{ror} é completamente baseada no paradigma \emph{Model View Controler} (MVC), facilitando a organização dos módulos de nossa arquitetura. Assim, a estrutura do código escrito em RoR é composta de componentes de Modelo, de Visão e de Controle. Os componentes de \textbf{modelo} correspondem aos dados - como eles são armazenados, obtidos, correlacionados. A parte de \textbf{visão} corresponde à parte gráfica da aplicação. Finalmente, os \textbf{controladores} realizam a manipulação de dados como um todo, e correspondem à parte lógica e funcional do código. Eles funcionam também como uma ponte entre modelo e visão, para que os dados transitem em ambos os sentidos.
%
% Como \emph{framework}, foi utilizado o Ruby on Rails (RoR),  por ser baseado no paradigma \emph{Model View Controler}, que facilita a divisão necessária para os módulos da arquitetura. 

Considerando o RoR, o submódulo analizador de tráfego (AT) da arquitetura corresponde a um controlador. Assim, uma requisição à aplicação será interceptada por esse componente de controle, que realizará a medição de estatísticas, e imediatamente acionará o controlador que corresponde ao funcionamento da aplicação em si. Deve-se notar, contudo, que o tempo despendido neste controlador é ínfimo. 
%
A Figura~\ref{fig:dfd} apresenta um fluxograma da implementação realizada. 
%
Em outras implementações, caso se perceba que o tempo afeta o funcionamento do mecanismo de mitigação, este processamento poderia ainda ser realizado em segundo plano. 

\begin{figure}[ht!]
	\centering
	\includegraphics[width=0.84\textwidth]{images/dfd-3.pdf}
	% \hskip 1cm
	\caption{Operações da implementação para a mitigação}
	\label{fig:dfd}
\end{figure}

Quando o AT detectar a existência de um possível ataque, uma nova instância \emph{cloud} é criada pelo submódulo INA e a aplicação é replicada para esta instância, paralisando a aplicação original, que passa a responder apenas como redirecionador. O processo de reinstanciação da aplicação na implementação realizada consiste da existência prévia de uma segunda aplicação, inicialmente sem nenhum recurso alocado. %Entretanto, embora esta instância estaria pronta para que seus processos sejam escalados assim que necessário, esta abordagem não se comporta adequadamente no cenário de reinstanciação recursiva. A segunda abordagem envolve a hospedagem do projeto em um repositório do GitHub, que poderá ser clonado para a especificação da segunda instância a partir do código Ruby. 

Uma particularidade interessante do \emph{framework} RoR é a existência de um arquivo de rotas. A implementação do submódulo redirecionador de tráfego (RT) é realizada em cima deste arquivo, chamado \emph{routes.rb}. Para a exibição de qualquer página dinâmica da aplicação, o arquivo de rotas é inevitavelmente chamado. Desta forma, ele é utilizado para a adição de clientes na \emph{blacklist} e respectiva filtragem dos clientes bloqueados pelo gerenciador da \emph{blacklist} (GB). No redirecionamento do tráfego para uma nova instância, uma entrada será adicionada, bloqueando o cliente em questão por determinado tempo.

A \emph{blacklist} em si e as diversas outras variáveis de controle são gerenciadas pela base de dados em \emph{cloud}~\cite{redis}. Esta base de dados é conhecida por sua simplicidade e eficiência. Ela basicamente mapeia \textbf{chave} e \textbf{dado}, oferecendo tempos de escrita e de leitura correspondentes à \emph{hashing}. A implementação da \emph{blacklist} foi feita utilizando o endereço IP de um cliente como chave, e o tempo que este cliente permanecerá bloqueado como dado. Por ser, indiretamente, um mecanismo de \emph{hashing}, o tempo de busca por um cliente será O(1), o que é excelente para um mecanismo que filtrará todo tráfego que chega à aplicação. %Uma visão estruturada das funções realizadas pela arquitetura proposta é descrita no diagrama de fluxo ilustrado na Figura~\ref{fig:dfd}.

%Por fim, um aspecto interessante do uso do Heroku são os diversos \emph{addons} já customizados para o uso nele. Em particular, existe um \emph{addon} chamado \emph{New Relic} que é designado à coleta de diversas métricas para a análise de desempenho. Com seu uso, será possível saber com precisão o que se passa em todas as instâncias \emph{cloud} de uma perspectiva diretamente interna à esta \emph{cloud}. Assim, poderemos coletar dados não só da perspectiva externa à \emph{cloud}, perspectiva de usuário, como também, de perspectiva interna~à~ela.

\section{Evaluation}
%!TEX root = ./proposta3.tex


% Uma solução para mitigar ataques pode ser avaliada quanto a sua capacidade de detectar ataque, assim como, a de reagir a ele. Outra abordagem para avaliar uma solução é quanto a sua capacidade em manter as condições normais de funcionamento do cenário, mesmo sob ataque. Segundo \cite{4600003} é importante para um sistema de defesa estimar diversos aspectos como: custo de desenvolvimento, desempenho, degradação do serviço e custo de robustez. 
% 
% A maioria das métricas para calcular o impacto de ataques DDoS estão relacionadas com as medidas de eficiência dos padrões de defesa \cite{4809152}. Atualmente, são consideradas estratégias de medição da quantidade de tráfego legítimo que chega até a aplicação. Outros trabalhos tem se concentrado na medida do tempo de resposta para avaliar a eficiência de uma solução. \cite{Mirkovic:2007:TUM:1281700.1281708} utiliza um modelo baseado em \emph{threshold} como métrica para aferir o impacto de DDoS. Quando uma medida excede este \emph{threshold}, ocorre a  indicação da baixa qualidade do serviço. Esta medida é indicada para aplicações fim-a-fim, como o HTTP.

A avaliação da nossa arquitetura consiste principalmente na análise
da capacidade do servidor em atender novas requisições, sendo que o atendimento pode consistir apenas no redirecionamento. 
Se o ataque de DDoS for devidamente
mitigado, as requisições de atacantes serão ignoradas, após a inclusão do requisitante na \emph{blacklist}. Assim, o servidor na \emph{cloud} deverá ser capaz de redirecionar apenas clientes legítimos 
para a nova instância e garantirá que eles terão acesso direto nas próximas requisições. 
%
% Portanto, de acordo com as métricas especificadas por \cite{4600003}, este trabalho de pesquisa foi avaliado pelo uso de métricas como o tempo de resposta do servidor para requisições atendidas, a taxa de requisições atendidas em relação ao número de clientes e a carga gerada pelos módulos da arquitetura de acordo com o número de clientes \footnote{O termo ``clientes'', usado neste parágrafo engloba tanto clientes legítimos quanto atacantes.}. 
%
% Cabe ressaltar que este trabalho não necessita de uma precisão muito grande na detecção do tipo de ataque, no sentido de que é melhor realizar uma calibragem muito sensível e possuir falsos positivos do que possuir falsos negativos, isto se deve à natureza do mecanismo implementado. Se uma nova instância for criada e o tráfego for redirecionado à toa, o custo será de apenas alguns milisegundos de latência. Caso o mecanismo não detecte um ataque, o custo será muito mais significativo. Desta maneira, a avaliação quanto à falsos positivos não é crucial. % quanto a avaliação não detecção do ataque. 



% \begin{figure}[t!]
% 	\centering
% 	\includegraphics[height=8cm]{images/overhead-p1.pdf}
% 	\includegraphics[height=8cm]{images/overhead-p2.pdf}
% 	\caption{Funções da arquitetura de mitigação de DDoS}
% \end{figure}


\subsection{Scenarios}
%!TEX root = ./proposta3.tex

Para a simulação do tráfego necessário, podem ser utilizadas ferramentas que enviam requisições HTTP à aplicação. Como ferramentas, podem ser citados os comandos \emph{curl} e \emph{ab} e a aplicação \emph{LOIC}. A diferença fundamental dentre elas é o nível na qual operam. Enquanto o comando \textbf{curl} opera realizando instâncias singulares de operações simples como \emph{GET}s e \emph{PUT}s, o comando \textbf{ab} automatiza o processo realizando diversas operações de acordo com alguns parâmetros. É possível customizar o nível de concorrência e o intervalo entre as requisições, por exemplo. Tal ferramenta possibilita a medição de algumas métricas como a taxa de entrega de pacotes e o tempo de resposta. Em um nível ainda maior, a aplicação~\cite{loic} foi desenvolvida a fim de realizar ataques de DoS\footnote{Diversos clientes a utilizam a fim de gerar um ataque de DDoS.}. Ela realiza automaticamente a calibragem de diversos parâmetros em níveis inferiores. Entretanto, ela ainda permite a customização de alguns, como a porta de ataque e o protocolo a ser utilizado.\\

Considerando estas ferramentas, alguns cenários foram elaborados para a avaliação. 
%
Para a simulação de clientes, o comando \emph{curl} foi utilizado para criar um \emph{script} que atua como um cliente legítimo ou atacante. O comando requisita a página em questão através de um \emph{GET}. Caso a resposta indique uma mudança de endereço, o \emph{script} é responsável por seguir todas as mudanças e redirecionamentos com chamadas subsequentes do comando \emph{curl}, até que o destino final seja de fato atingido. Deve-se ressaltar que esse comportamento é dificultado em atacante de DDoS, pois eles perderiam muito a sua eficiência ao aguardar por respostas de requisições, analisá-las, e seguir até o destino final.

Para a obtenção de tempos de resposta e taxa de perda de pacotes da perspectiva do cliente, o comando \emph{ab} pode ser empregado para realizar as medições. Diversas faixas de parâmetros podem ser estipuladas para simular diferentes tipos de comportamentos ou cenários. O uso do \emph{ab} torna-se preferível ao do \emph{curl} quando o objetivo é coletar métricas e não simular um cliente genuíno, afinal, o comando \emph{ab} medirá o desempenho de instâncias \emph{cloud} isoladas, sem realizar qualquer redirecionamento de tráfego. Este aspecto do comportamento do \emph{script} em questão é importante para a diferenciação entre ataques DDoS e \emph{Flash Crowds}~\cite{Thapngam:2011p27061}. Enquanto um ataque é malicioso, uma \emph{Flash Crowd} indica que diversos clientes legítimos estão, de fato, realizando diversas requisições à aplicação. Este caso não será tratado, embora talvez pudesse ser identificado.\

\textbf{justificar o nao uso do LOIC e colocar o cenário}
Por fim, para a simulação do ataque DDoS em si, a ferramenta \emph{LOIC} pode ser utilizada, pois ela é uma ferramenta para ataques reais. Embora a escala em nosso experimento seja muito menor, a operação será baseada em um ataque real e, portanto, condizente com a realidade. Para a obtenção de diferentes endereços IP, diferenciando um DoS de um DDoS, a ferramenta será utilizada simultaneamente por computadores de endereços diferentes.
No cenário de simulação foram utilizados oito nós atacantes executando \emph{scripts} de ataques simultâneos, originando 25, 50, 75 e 100 processos atacantes sobre uma latência de 3 a 7 milisegundos.



\subsection{Results}
%!TEX root = ./proposta3.tex

Para explorar e validar a proposta deste trabalho foram realizados experimentos e os resultados analisados sobre aspectos tempo de resposta e \emph{overhead}. Primeiramente, foi avaliado o impacto da mitigação de DDoS no tempo de resposta com relação à dimensão do ataque e, em seguida, o \emph{overhead} introduzido pelo mecanismo.

quantidade de páginas recebidas com sucesso, só temos perdas (por causa de timeouts) quando não usamos a BlackList e o servidor tenta responder mandando a página pro atacante, que descarta ela sem nem ver e continua atacando


foi analisado o \emph{overhead} causado pela arquitetura proposta em relação a execução normal da aplicação. O gráfico \ref{fig:overhead} demonstra que a solução causa um impacto insignificante na execução 
%\begin{figure}[h!]
%\centering
%\includegraphics[width=0.55\textwidth]{images/overhead.eps}
%\caption{Impacto daTempo de resposta para clientes legítimos}
%\label{fig:imp_sem_black}
%\end{figure}

Primeiramente foi analisado o impacto que um ataque de DDoS pode causar no cenário de avaliação, conforme os dados apresentados no gráfico \ref{fig:imp_sem_black} que mostra o tempo de resposta em segundos para clientes legítimos

%\begin{figure}[h!]
%\centering
%\includegraphics[width=0.55\textwidth]{images/imp_sem_black.eps}
%\caption{Tempo de resposta para clientes legítimos}
%\label{fig:imp_sem_black}
%\end{figure}

http://cl.ly/1m1m333S0S3d443x3c3P -> overhead



http://cl.ly/2i3t0d131c1R3O433e1K -> tempo de resposta em segundos para clientes legítimos, legenda eixo X (25/50/75/100) indica quantos processos de ataque DDoS sao executados SIMULTANEAMENTE em cada atacante (sao 8 atacantes)



http://cl.ly/1w0c383v3w3j0u01391L -> quantidade de páginas recebidas com sucesso, só temos perdas (por causa de timeouts) quando não usamos a BlackList e o servidor tenta responder mandando a página pro atacante, que descarta ela sem nem ver e continua atacando




\section{Conclusion}
%!TEX root = ./proposta3.tex

Este trabalho apresentou uma proposta de arquitetura de mitigação para ataques de DDoS direcionados à aplicações web hospedadas em \emph{cloud}. A arquitetura é dependente apenas da existência do ambiente na \emph{cloud}. Os quatro sub módulos especificados garantem o funcionamento das etapas de forma autônoma até mesmo para gerar uma nova instância da aplicação. Assim como, permite livre acesso para os clientes legítimos, dado que apenas eles realizarão o redirecionamento solicitado. O uso da \emph{blacklist} tem gerencimento eficiente pelo uso de tempo de validade, que no caso de atacantes, terá aumento exponencial para reincidências.

Os resultados alcançados nas experimentações demonstram a validade da proposta de arquitetura de mitigação a ataques de DDoS pois, não onera financeiramente o fornecedor da aplicação, apresenta baixo \emph{overhead} ao ser incluído no ambiente da aplicação e consegue direcionar o tráfego legítimo de modo satisfatório, impossibilitando os atacantes de acessarem a nova instância criada.
% Outra caractística considerável da arquitetura é a tolerância a falhas mesmo em cenários de ataques, assim como a o baixo tempo de resposta as requisções e nenhuma perda na taxa de entrega.   

% A arquitetura proposta trabalha a filtragem e redirecionamento de requisições legítimas para uma nova aplicação sem incidência de ataques, porém ela considera que a aplicação original está sobre ataque pela quantidade de tráfego que recebe. Mecanismos mais robustos para a detecção de DDoS podem ser desenvolvidos como trabalhos futuros complementando a solução já implementada, garantindo que esta apresente uma alta acurácia na detecção de tipos específicos de ataque de DDoS. Outra proposta de trabalhos futuros diz respeito a implementação e checagem da \emph{blacklist} em níveis mais baixos e otimizados como em servidores HTTP.


% \newpage
% \bibliographystyle{unsrt}

\bibliographystyle{templates/sbc}
\bibliography{parts/proposta}
\end{document}
