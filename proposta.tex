\documentclass[a4paper, 11pt]{article}

\usepackage{graphicx}
\usepackage[brazil]{babel}   
\usepackage[utf8]{inputenc}  
\usepackage{epsfig}
\usepackage{float}
\usepackage{graphics}
\usepackage{url}
\usepackage[tight,footnotesize]{subfigure}
\usepackage{stfloats}	
\usepackage{enumerate}
\usepackage[left=3cm,top=3cm,right=3cm,bottom=3cm]{geometry}

% correct bad hyphenation here
\hyphenation{op-tical net-works semi-conduc-tor}


\begin{document}


{
\begin{center}
{\LARGE \textbf{Projeto da disciplina de Gerência de Redes}}
\vskip 0.5cm
{\Large Cinara, Fernando Cezar, Fernando Gielow, Nadine}
\end{center}
}

\section{Proposta inicial}

% introduzir ataques DDoS e a importância do problema
Um ataque DDoS consiste, genericamente, da tentativa de tornar indisponíveis os recursos oferecidos por alguma entidade [ref]. Na internet, este tipo de ataque tem se tornado cada vez mais comum, visando \emph{websites} de grandes empresas. Recentemente, houveram ataques aos \emph{websites} da Amazon, do Paypal e da bandeira de cartões de crédito Visa [ref], que acarretaram em grandes prejuízos à estas empresas.

Diversas dificuldades são encontradas ao se tentar mitigar os efeitos destes ataques. Em servidores de hospedagem tradicionais, os recursos são limitados e, assim, quando o número de requisições ultrapassa um patamar máximo, não há como continuar respondendo efetivamente as requisições. Em abordagens mais avançadas, como é o caso de hospedagem em \emph{clouds} [ref], tais ataques acarretam na alocação de uma quantidade imensa de recursos, a afim de tratar todas as requisições que chegam ao servidor, independente de serem clientes legítimos ou atacantes. Embora esta abordagem consiga tratar ataques DDoS até dados limites, é uma abordagem custosa, pois todos os recursos utilizados nesta tentativa de mitigação serão cobrados [sPoW].

O trabalho [xx] tenta tratar estes ataques através da criação de uma nova instância da aplicação. Uma vez que um ataque DDoS é detectado, ele tenta detectar os atacantes através de um PING - caso o possível atacante não responda o PING, ele é considerado como um atacante. Desta maneira, apenas os clientes que responderem o PING serão redirecionados para a nova instância da aplicação. Entretanto, tal abordagem depende da premissa que atacantes jamais responderão a PINGs e que clientes genuínos sempre responderão, o que nem sempre condiz com a realidade. 

% objetivo
O objetivo deste trabalho é apresentar uma proposta de um esquema de mitigação de ataques DDoS para aplicações hospedadas em \emph{clouds}. Este esquema será composto pelos seguintes módulos: (i) detecção de comportamento similar a ataques DDoS; (ii) redirecionador de tráfego; (iii) gerenciador de \emph{blacklist}. O módulo (i) observará proativamente o padrão de tráfego de entrada, visando detectar um ataque DDoS. Caso um ataque seja detectado, será criada uma nova instância da aplicação em uma outra instância de servidor \emph{cloud}, consequentemente com um endereço IP diferente. Desta forma, o módulo (ii) será responsável por tratar todo o tráfego de entrada, informando o endereço do novo servidor, e inserindo o IP origem da requisição à uma \emph{blacklist}, através do módulo (iii). Desta maneira, o atacante não afetará a nova instância da aplicação, pois ele não interpretará as respostas que informam o IP da nova instância. 

Estamos estudando a implementação de tal mecanismo, sendo que ele possivelmente será implementado e testado nos serviços \emph{cloud} disponibilizados pela Amazon [ref]. Serão apresentados resultados que demonstram a eficiência do mecanismo em tratar ataques DDoS de pequenos portes.

% [xx] = Securing cloud from DDOS Attacks using Intrusion Detection System in Virtual Machine
\end{document}