\documentclass[a4paper, 11pt]{article}

\usepackage{graphicx}
\usepackage[brazil]{babel}   
\usepackage[utf8]{inputenc}  
\usepackage{epsfig}
\usepackage{float}
\usepackage{graphics}
\usepackage{url}
\usepackage[tight,footnotesize]{subfigure}
\usepackage{stfloats}	
\usepackage{enumerate}
\usepackage[left=3cm,top=3cm,right=3cm,bottom=3cm]{geometry}

% correct bad hyphenation here
\hyphenation{op-tical net-works semi-conduc-tor}


\begin{document}


{
\begin{center}
{\LARGE \textbf{Arquitetura para Mitiga\c{c}\~{a}o de Ataques DDoS}}
\vskip 0.5cm
{\Large Cinara, Fernando Cezar, Fernando Gielow, Nadine}
\end{center}
}

\section{Proposta Inicial}

% introduzir ataques DDoS e a importância do problema
Um ataque DDoS (\emph{Distributed Denial-of-Service})
 consiste, genericamente, da tentativa de tornar indisponíveis
os recursos oferecidos por alguma entidade \cite{Zhang:11}. Na Internet, este
tipo de ataque tem se tornado cada vez mais comum, visando \emph{websites} de
grandes empresas. Recentemente, houveram ataques aos \emph{websites} da Amazon,
do Paypal e da bandeira de cartões de crédito Visa \cite{Zuckerman:10,ddosatks}, que
acarretaram em grandes prejuízos à estas empresas.

Diversas dificuldades são encontradas ao se tentar mitigar os efeitos destes
ataques. Em servidores de hospedagem tradicionais, os recursos são limitados e,
assim, quando o número de requisições ultrapassa um patamar máximo, não há como
continuar respondendo efetivamente às requisições. Em abordagens mais avançadas,
como é o caso de hospedagem em \emph{clouds} \cite{Zhang:10}, tais ataques
acarretam na alocação de uma quantidade imensa de recursos, as vezes até mesmo capazes de
suprir todas as requisições que chegam ao servidor, independente de serem
clientes legítimos ou atacantes. Embora esta abordagem consiga tratar ataques
DDoS para determinados limites, é uma abordagem custosa, pois todos os recursos
utilizados nesta tentativa de mitigação acarretar\~ao em custos
\cite{Soon:10}.

O trabalho \cite{Bakshi:10} prop\~oe tratar este cen\'ario de ataques através da
criação de uma nova instância da aplicação. Uma vez que um ataque DDoS é
detectado, a proposta busca detectar os atacantes através de PINGs -
caso um cliente suspeito de ser atacante não responder ao PING, ele é considerado como um
atacante, de fato. Desta maneira, apenas os clientes que responderem ao PING serão
redirecionados para a nova instância da aplicação. Entretanto, tal abordagem
depende da premissa que atacantes jamais responderão a PINGs e que clientes
genuínos sempre responderão, o que nem sempre condiz com a realidade. 

\section{Objetivo}
Este trabalho de pesquisa gerar\'a uma arquitetura para mitigação de ataques de
DDoS executados sobre uma aplicação hospedada em uma \emph{cloud}. Esta
arquitetura, dever\'a monitorar o tr\'afego da aplica\c{c}\~{a}o e quando
detectar
ocorrência de ataque de DDoS, ser\'a respons\'avel por instanciar uma nova instância desta
aplica\c{c}\~{a}o, garantindo que nenhum tr\'afego malicioso a alcance.

Esta arquitetura será composta pelos seguintes módulos:
\begin{enumerate}[i]
  \item detecção de comportamento similar a ataques DDoS;
  \item redirecionador de tráfego;
  \item gerenciador de \emph{blacklist};
  \item instanciador de nova aplica\c{c}\~{a}o.
\end{enumerate}  

O módulo (i) observará proativamente o padrão de tráfego de entrada, visando
detectar um ataque DDoS. Caso um ataque seja detectado, o m\'odulo (iv) criará uma nova
instância da aplicação em uma outra instância de servidor no \emph{cloud}, consequentemente com um endereço IP diferente.
Com isso, o módulo (ii) será responsável por tratar todo o tráfego de
entrada, informando o endereço do novo servidor, e inserindo o IP origem da
requisição à uma \emph{blacklist}, através do módulo (iii). Assim, o
atacante não afetará a nova instância da aplicação, pois ele não interpretará as
respostas que informam o IP da nova instância.


\section{Metodologia}
A pesquisa empregar\'a metodologia baseada em implementa\c{c}\~{a}o e medidas de
efici\^encia da solu\c{c}\~{a}o, que se propõem a mitigar o impacto causado pelo
tipo de ataque DDoS. Para que os objetivos sejam atingidos, a metodologia dever\'a
compreender etapas como pesquisa bibliogr\'afica para levantamento de trabalhos
correlatos. Pela avalia\c{c}\~{a}o destes trabalhos ser\'a poss\'ivel
classificar as
solu\c{c}\~oes aplicadas tanto para detectar como para controlar ataques de
DDoS,
especificamente para ambientes de \emph{cloud}. Tendo como refer\^encia as
solu\c{c}\~oes aplicadas e os problemas ainda existentes, nosso trabalho
buscar\'a
alternativas capazes de efetivamente trazer melhorias ao problema levantado,
de acordo com a especificação da seção anterior.
Estas melhorias ser\~ao aplicadas em um ambiente real de \emph{cloud} como
Heroku~\cite{heroku}, Amazon~\cite{amazon}
ou Linode~\cite{linode} e monitoradas quanto a sua efici\^encia, considerando a
execu\c{c}a\~o correta da aplica\c{c}\~{a}o e o redirecionamento de apenas requisições
confi\'aveis.


\section{Proposta de Solu\c{c}\~{a}o}
Inicialmente, nossa proposta compreender\'a o uso de tratamento de tr\'afego
para identifica\c{c}\~{a}o de 

\section{Avalia\c{c}\~{a}o}
A etapa de avalia\c{c}\~{a}o da proposta ser\'a realizada por monitoramento do 

% [xx] = Securing cloud from DDOS Attacks using Intrusion Detection System
%in Virtual Machine
\section{Etapas da Proposta}
\begin{itemize}
 \item Pesquisa
  \begin{itemize}
  \item Levantamento de trabalhos relacionados
  \item Estudo e comparativo de  mecanismos de mitiga\c{c}\~{a}o para
DDoS em diversas arquiteturas 
  \end{itemize}
 \item Implementa\c{c}\~{a}o
  \begin{itemize}
    \item %placeholder
  \end{itemize}
 \item Teste de Desempenho 
  \begin{itemize}
    \item %placeholder
  \end{itemize}
\end{itemize}

\section{Por que DDoS?}



\newpage
\bibliographystyle{unsrt}
\bibliography{proposta}
\end{document}