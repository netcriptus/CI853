%!TEX root = ./proposta3.tex

Para explorar e validar a proposta deste trabalho foram realizadas avaliações em um cenário composto 

Nesta seção, a solução proposta é analisada sobre ??? aspectos. 

Primeiramente, foi analisado o \emph{overhead} causado pela arquitetura proposta em relação a execução normal da aplicação. O gráfico \ref{fig:overhead} demonstra que a solução causa um impacto insignificante na execução 
%\begin{figure}[h!]
%\centering
%\includegraphics[width=0.55\textwidth]{images/overhead.eps}
%\caption{Impacto daTempo de resposta para clientes legítimos}
%\label{fig:imp_sem_black}
%\end{figure}

Primeiramente foi analisado o impacto que um ataque de DDoS pode causar no cenário de avaliação, conforme os dados apresentados no gráfico \ref{fig:imp_sem_black} que mostra o tempo de resposta em segundos para clientes legítimos

%\begin{figure}[h!]
%\centering
%\includegraphics[width=0.55\textwidth]{images/imp_sem_black.eps}
%\caption{Tempo de resposta para clientes legítimos}
%\label{fig:imp_sem_black}
%\end{figure}

http://cl.ly/1m1m333S0S3d443x3c3P -> overhead



http://cl.ly/2i3t0d131c1R3O433e1K -> tempo de resposta em segundos para clientes legítimos, legenda eixo X (25/50/75/100) indica quantos processos de ataque DDoS sao executados SIMULTANEAMENTE em cada atacante (sao 8 atacantes)



http://cl.ly/1w0c383v3w3j0u01391L -> quantidade de páginas recebidas com sucesso, só temos perdas (por causa de timeouts) quando não usamos a BlackList e o servidor tenta responder mandando a página pro atacante, que descarta ela sem nem ver e continua atacando


