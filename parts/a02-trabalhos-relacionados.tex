%!TEX root = ./proposta.tex


As pesquisas que envolvem propostas de mitigação de DDoS em arquiteturas de \emph{cloud}, ainda são consideradas incipientes e distantes de uma convergência. Dentre as poucas propostas para estes ambientes, destaca-se o \emph{framework} pró-ativo CluB, apresentado em \cite{Hazelhurst:2008:SCU:1456659.1456671}, que %considera uma \emph{cloud} como uma rede constituída de um conjunto de \emph{clusters} ou AS. Este trabalho 
sugere
que sejam selecionados determinados roteadores, dispostos de forma distribuída, para a análise de tráfego e consequente prevenção de atividade maliciosa. %que as requisições maliciosas alcancem a aplicação. %Estes roteadores são responsáveis por gerar \emph{tokens} de autenticação para legitimar os pacotes, sendo que a autenticação é necessária para a entrada, saída ou trânsito na arquitetura. Cada \emph{cluster} tem seu código de autenticação, que é trocado periodicamente, podendo ser gerado por uma função \emph{hash}, como MD5 ou SHA. O uso de ferramentas apropriadas de criptografia e atualizações periódicas de componentes da infraestrutura fazem parte da proposta do CluB. 
Neste \emph{framework}, todo pacote %, malicioso ou não,
 deve ser verificado para entrar, sair ou transitar na arquitetura. Cada roteador alocado realiza a verificação, o que é custoso devido ao \emph{overhead} causado pela autenticação de cada pacote e pela necessidade inviável de alterar o comportamento dos roteadores. %Também é necessária a implantação e atualização dos algoritmos de análise de tráfego na arquitetura onde estaria sendo utilizado o CluB. Esta questão se torna inviável ao se tratar de uma \emph{cloud}, devido à nebulosidade de sua arquitetura e infraestrutura.

Em \cite{Verkaik:2006:PCD:1162666.1162673}, é apresentado um esquema pró-ativo que emprega Comunidades de Interesse (COIs) para capturar dados sobre o comportamento coletivo das entidades remotas, utilizando-os para predizer o comportamento futuro. Tal esquema assume que os clientes que tiveram relações legítimas anteriormente possuem bons indícios e podem ser considerados novamente legítimos. %Estas afirmações são geradas da observação de comunicações normais da rede e são utilizadas em conjunto com políticas específicas do servidor para mitigar pró-ativamente os ataques de DDoS, usando mecanismos existentes nos roteadores. 
%
Entretanto, a identificação dos clientes antigos não é tão trivial. Além do \emph{overhead} gerado pela verificação, os endereços IPs são normalmente dinâmicos e a exigência da realização de \emph{login} para a identificação não é possível, dado que o ataque de DDoS pode impossibilitar uma operação de identificação.
%
%
Em \cite{Bakshi:10}, os ataques são tratados através da criação de uma nova instância da aplicação. Uma vez que um ataque DDoS é detectado, o mecanismo proposto busca identificar os atacantes através de PINGs: caso um cliente suspeito de ser atacante não responda ao PING, ele é considerado como um atacante, não sendo redirecionado para a nova instância da aplicação. 
%. Assim, apenas os clientes que respondem ao PING são redirecionados para a nova instância da aplicação. 
Entretanto, essa solução assume que sempre e apenas clientes genuínos responderão a PINGs, o que as vezes não condiz com a realidade.



% \cite{Walfish:2010:DDO:1731060.1731063} apresenta uma forma de mitigação de ataque classificada como defesa baseada em recursos \cite{Dwork:1992:PVP:646757.705669}.
% Toda vez que um determinado limite de banda é consumido com requisições para um servidor, este servidor, antes que seus recursos se esgotem, encoraja seus clientes a enviar volumes ainda mais altos de tráfego. Considera-se que os atacantes já estariam usando sua capacidade máxima e, assim, eles não poderiam reagir ao encorajamento. A proposta se baseia na premissa que bons clientes têm condições de aumentar seu uso de banda e reagir de forma drástica ao encorajamento. O resultado pretendido é que os bons clientes dominem os maus clientes ao capturar uma fração maior de recursos do servidor. O cliente será atendido caso ele tenha banda o suficiente para se sobressair mediante o tráfego dos atacantes. Um tanto curiosa, esta proposta ocasiona diversos problemas como
% o encorajamento a recebimento de ainda mais tráfego em cenários de ataque. 
% Difícilmente um serviço conseguirá atender a tantas requisições e clientes legítimos não necessariamente dominarão o tráfego que chega ao servidor.


% \hyphenation{i-den-ti-fi-ca-ção}

A eficácia desses esquemas de mitigação depende diretamente da capacidade de identificação ou filtragem dos clientes legítimos. 
%
A solução WebSoS \cite{Stavrou:2005:WOS:1090583.1648614} % após a detecção do ataque. 
oferece uma filtragem robusta de tráfego atacante e bloqueio de requisições~não aprovadas, formando assim um \emph{overlay} seguro que mitiga DDoS em servidores web. O servidor utiliza mecanismos de autenticação criptográfica e um teste gráfico de Turing \cite{Dietrich00analyzingdistributed} para diferenciar clientes humanos de \emph{scripts} de ataque. Estes procedimentos, segundo os testes dos autores, não sobrecarregam o funcionamento do serviço, porém exigem que os roteadores localizados no perímetro do servidor sejam reconfigurados, o que é inviável para arquiteturas de \emph{cloud}.
