%!TEX root = ./proposta.tex

A necessidade de proteger ou mitigar as arquiteturas de rede de ataques de DDoS tem sido reconhecida tanto no meio acadêmico quanto comercial. Segundo \cite{1039856}, três seriam as linhas clássicas de defesa contra ataques de DDoS, compreendendo a \textbf{prevenção}, a \textbf{reação}, ou ambas, na chamada defesa \textbf{híbrida}. A prevenção tenta eliminar a possibilidade de ataques de DDoS, isto é, evita a negação de serviço para os clientes legítimos. A linha de reação detecta o ataque e responde imediatamente, e a abordagem híbrida combina os métodos anteriores, não só prevenindo mas também reagindo à ataques. A maioria dos sistemas de segurança são construídos focados na prevenção de ataques \cite{4429182}. As abordagens clássicas buscam reduzir os riscos a zero, o que é impraticável para todos os tipos de riscos, pois podem ser custosos e complexos.

Atualmente, os sistemas tem se tornado tão complexos que é impossível identificar e corrigir todas as vulnerabilidades antes que elas se tornem ataques ou intrusões e, muitas vezes, impossível de se recuperar de falhas decorrentes. Além disso, a garantia de funcionamento destes sistemas sob condições de ataques ou intrusões têm se tornado uma prioridade. Diversos autores como \cite{Verissimo},
 \cite{4796927} e \cite{1424871}, têm defendido que as linhas clássicas de defesa não apresentam abordagens que garantam a tolerância a falhas em caso de ataques ou intrusões.
  
 Assim, uma nova abordagem foi definida chamada de \textbf{tolerância a intrusão}, a qual lida com sistemas de segurança que garantem o funcionamento dos sistemas mesmo que estejam sob problemas de segurança, minimizando ao máximo os prejuízos atéo retorno ao fluxo normal de funcionamento \cite{Fraga_Powell_1985}. Ao invés de tentar se prevenir de todo tipo de ataque ou intrusão, os sistemas passam a tolerar os problemas simples de segurança e criam mecanismos de controle à intrusão  de forma que eles não causem nenhuma falha ao sistema. Esta abordagem faz uso de replicação ou redundância para garantir os aspectos de tolerância e são aplicados,especificamente, quando a forma de intrusão ou ataque seja diferenciada ou desconhecida. 

Embora os ataques de DDoS ainda persistam e tenham crescido ultimamente, dadas as capacidades das redes, este tipo de ataque pode ser considerado comum e bem conhecido pelos sistemas de segurança já existentes executando em arquiteturas comuns portanto, o seu tratamento pode ser incluído dentro da abordagem clássica. Assim, este trabalho de pesquisa propõe uma arquititetura de mitigação classificada como reativa, pois verifica o tráfego e reage as suas anomalias, e tolerante a falhas por garantir o funcionamento da aplicação mesmo sob ataque, usando uma forma de replicação da aplicação, embora a aplicação original seja mantida apenas como redirecionador e não mais com suas funcionalidades originais.


As pesquisas que envolvem propostas de mitigação de DDoS em arquiteturas de \emph{cloud}, ainda são consideradas incipientes e distantes de uma convergência. Dentre as poucas propostas para estes ambientes, destaca-se o \emph{framework} pró-ativo CluB, apresentado em \cite{Hazelhurst:2008:SCU:1456659.1456671}, que considera uma \emph{cloud} como uma rede constituída de um conjunto de \emph{clusters} ou AS. Este trabalho sugere
que sejam selecionados determinados roteadores, dispostos de forma distribuída, para análise de tráfego e consequente prevenção de que as requisições maliciosas alcancem a aplicação. Estes roteadores são responsáveis por gerar \emph{tokens} de autenticação para legitimar os pacotes, sendo que a autenticação é necessária para a entrada, saída ou trânsito na arquitetura. Cada \emph{cluster} tem seu código de autenticação, que é trocado periodicamente, podendo ser gerado por uma função \emph{hash}, como MD5 ou SHA. O uso de ferramentas apropriadas de criptografia e atualizações periódicas de componentes da infraestrutura fazem parte da proposta do CluB. Neste \emph{framework}, todo pacote, malicioso ou não, precisa ser verificado para entrar, sair ou transitar na arquitetura. Cada roteador alocado  deverá realizar a verificação, o que é custoso devido ao \emph{overhead} causado pela autenticação de cada pacote. Também é necessária a implantação e atualização dos algoritmos de análise de tráfego na arquitetura onde estaria sendo utilizado o CluB. Esta questão se torna inviável ao se tratar de uma \emph{cloud}, devido à nebulosidade de sua arquitetura e infraestrutura.

\cite{Verkaik:2006:PCD:1162666.1162673} apresentam outra proposta pró-ativa, que emprega Comunidades de Interesse (COIs) para capturar dados sobre o comportamento coletivo das entidades remotas, utilizando-os para predizer o comportamento futuro. Tal proposta se baseia no fato de que clientes que tiveram relacionamentos legítimos anteriormente possuem bons indícios e podem ser considerados novamente legítimos. Estas afirmações são geradas da observação de comunicações normais da rede e são utilizadas em conjunto com políticas específicas do servidor para mitigar pró-ativamente os ataques de DDoS, usando mecanismos existentes nos roteadores. 
%
Entretanto, a identificação dos clientes passados não é tão trivial. Além do pequeno \emph{overhead} gerado pela verificação, endereços IPs são normalmente dinâmicos e a exigência da realização de \emph{login} para a identificação não é possível, dado que o ataque de DDoS pode a impossibilitar.


Em \cite{Bakshi:10}, os ataques são tratados através da criação de uma nova instância da aplicação. Uma vez que um ataque DDoS é detectado, a proposta busca identificar os atacantes através de PINGs: caso um cliente suspeito de ser atacante não responda ao PING, ele é considerado como um atacante, de fato. Desta maneira, apenas os clientes que responderem ao PING serão 
redirecionados para a nova instância da aplicação. Entretanto, tal abordagem depende da premissa que atacantes jamais responderão a PINGs e que clientes genuínos sempre responderão, o que nem sempre condiz com a realidade.



%Filtro de Bloom

\cite{Walfish:2010:DDO:1731060.1731063} apresenta uma forma de mitigação de ataque classificada como defesa baseada em recursos \cite{Dwork:1992:PVP:646757.705669}. %A mitigação emprega o procedimento de que 
Toda vez que um determinado limite de banda é consumido com requisições para um servidor, este servidor, antes que seus recursos se esgotem, encoraja seus clientes a enviar volumes ainda mais altos de tráfego. Considera-se que os atacantes já estariam usando sua capacidade máxima e, assim, eles não poderiam reagir ao encorajamento. A proposta se baseia na premissa que bons clientes têm condições de aumentar seu uso de banda e reagir de forma drástica ao encorajamento. O resultado pretendido é que os bons clientes dominem os maus clientes ao capturar uma fração maior de recursos do servidor. O cliente será atendido caso ele tenha banda o suficiente para se sobressair mediante o tráfego dos atacantes. Um tanto curiosa, esta proposta ocasiona diversos problemas como %a reação apenas quando o servidor já está sendo atacado e o procedimento 
o encorajamento a recebimento de ainda mais tráfego em cenários de ataque. 
% do que o próprio ataque poderia produzir, ou muitas vezes maior, 
Difícilmente um serviço conseguirá atender a tantas requisições e clientes legítimos não necessariamente dominarão o tráfego que chega ao servidor.


Obviamente, a eficácia de todos os esquemas depende criticamente da capacidade de se identificar ou filtrar os clientes legítimos. 
%
WebSoS \cite{Stavrou:2005:WOS:1090583.1648614} é uma adaptação de \emph{Secure Overlay Services} (SOS) \cite{Keromytis:2002:SSO:964725.633032} que mitiga DDoS em servidores web, reativamente após a detecção do ataque. Com uma filtragem robusta de tráfego e bloqueio de requisições~não aprovadas, forma-se um \emph{overlay} seguro. O servidor utiliza mecanismos de autenticação criptográfica e um teste gráfico de Turing \cite{Dietrich00analyzingdistributed} para diferenciar clientes humanos de \emph{scripts} de ataque. Estes procedimentos, segundo os testes dos autores, não sobrecarregam o funcionamento do serviço, porém exigem que os roteadores localizados no perímetro do servidor sejam configurados para controlar o tráfego, procedimento inviável para arquiteturas de \emph{cloud}.
