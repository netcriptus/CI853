%!TEX root = ./proposta3.tex

Diversas pesquisas têm sido desenvolvidas para tratar questões da Internet atual, que podem se propagar para a Internet do Futuro (IF). Tais problemas podem ser amplamente categorizados nas áreas de mobilidade, qualidade de serviço e segurança, os quais ainda caminham para soluções aceitáveis, agravados pelo surgimento de novas arquiteturas. Hoje tanto os dados quanto as aplicações são oferecidos em localizações físicas distintas e desconhecidas. Outra grande mudança ocorreu na forma de administrar um sistema, que antes era de âmbito mais local, com seus usuários e servidores característicos, e agora esses sistemas são hospedados em ambientes construídos pelo compartilhamento de recursos de diversos sistemas autônomos (AS) e heterogêneos~\cite{5486552}.

%O uso massivo dos recursos disponibilizados na Internet e toda a conectividade que este ambiente computacional proporciona com serviços para uso pessoal, comercial ou acadêmico, torna este ambiente um alvo visado para códigos maliciosos. Ademais, isso é agravado pela forma como a arquitetura TCP/IP pode favorecer um atacante. O protocolo IP (\emph{Internet Protocol}) omite as informações da verdadeira identidade de um emissor, essa não autenticação da fonte permite ao atacante realizar o ataque contra a vítima, podendo permanecer anônimo e impune~\cite{1039856}.

Apesar de muitos esforços em pesquisas, os ataques de \emph{Denial of Service} (DoS) ainda representam sérias ameaças a muitos servidores na Internet e se configuram como um dos principais desafios de segurança atualmente propagado para a IF, que interconectará muito mais dispositivos e indivíduos.  %Uma forma de se defender de ataques DoS é a prevenção e a reação. 
Um ataque DoS não visa invadir um computador para obter informações confidenciais, nem tão pouco alterar informações armazenadas nele. Seu objetivo é a indisponibilização de um serviço fornecido, utilizando-se do encaminhamento de grandes quantidades de tráfego ao hospedeiro do serviço. %Desta forma, este serviço não responderá a todas as requisições que lhe são encaminhadas. 
Essa questão torna-se ainda mais severa quando diversos geradores de tráfego intensificam o encaminhamento de tráfego de maneira distribuída, caracterizando um ataque de \emph{Distributed Denial of Service} (DDoS) \cite{Sachdeva08ddosincidents}. 
%
Embora tal carga seja um problema apenas momentâneo, em se tratando de aplicações destinadas ao comércio eletrônico, por exemplo, uma parada do serviço representa grandes perdas financeiras. 
%A consequência é o congelamento, a reinicialização, ou ainda o esgotamento completo de recursos necessários ao hospedeiro. Os serviços que mais sofrem com este tipo de ataque são aqueles que permitem requisições anônimas, como serviços web. Assim, o desafio de eliminar os ataques de DDoS está na dificuldade de determinar a diferença entre pacotes legítimos e pacotes de atacantes \cite{Li:2009:DDA:1683304.1684620}.

Com as novas arquiteturas de rede e de aplicações que configuram a Internet, têm surgido sistemas complexos e robustos como \emph{clouds} (nuvens), onde o desafio de mitigar ataques DoS torna-se ainda mais necessário. Embora a maioria das soluções comumente oferecidas para mitigar DDoS em \emph{cloud} se baseie na maior alocação de recursos \cite{Peng:2007:SND:1216370.1216373}, 
%
% porém elas também aumentam os custos do usuário para manter tais recursos. Este comportamento carateriza \emph{economic DDoS} (eDDoS) \cite{Soon:10}.  
%
% Assim, essas abordagens tornam-se inadequadas pois a premissa de alocação de recursos nem sempre é viável por ser custosa \cite{Bakshi:10}, \cite{Liu:2010:NFD:1866835.1866849}. 
%
%
essas abordagens tornam-se inadequadas pois a premissa da possibilidade de maior alocação de recursos
nem sempre é viável por ser custosa demais \cite{Bakshi:10}, \cite{Liu:2010:NFD:1866835.1866849}. Este comportamento carateriza o \emph{economic DDoS} (eDDoS) \cite{Soon:10}.
%
%
%
%devido a premissas que nem sempre viáveis %como supor que um cliente legítimo sempre responderá a PINGs, ou 
% por serem custosas demais \cite{Bakshi:10}, \cite{Liu:2010:NFD:1866835.1866849}. 
%Existem diversos registros de ataques que abalaram a Internet nos últimos tempos, como os ocorridos contra o Yahoo!, eBay, Amazon.com e diversos outros \emph{sites} populares em fevereiro de 2000.  
%No início de 2011 se observou o ataque sofrido pelo hospedeiro de \emph{blogs}, WordPress, que enfrentou o pior ataque de DDoS de sua história \cite{infoexame}.
% Ataques de DDoS são disparados em proporções alarmantes, de acordo com descobertas recentes que também revelam a engenharia aplicada que gera redes de zumbis. %Um exemplo destas é a rede TDL-4, que é classificada por especialistas em segurança como “não perfeitamente, mas praticamente indestrutível”, com aproximadamente 4,5 milhões de infecções só em 2011 \cite{tdl4}.
% A partir de fevereiro 2010, o grupo ativista \textit{hacker} conhecido como \textit{Anonymous} começou uma série de ataques de cunho político e ideológico contra várias instituições de porte internacional \cite{titstorm}. A parte mais massiva desses ataques era constituída de DDoS.
% \cite{infoexame}.  Assim, a mitigação de DDoS em \emph{clouds} ainda demanda pesquisas.

%A necessidade de proteger ou mitigar as arquiteturas de rede de ataques de DDoS tem sido reconhecida tanto no meio acadêmico quanto comercial. 
%Segundo \cite{1039856}, três seriam as linhas clássicas de defesa contra ataques de DDoS, compreendendo a \textbf{prevenção}, a \textbf{reação}, ou ambas, na chamada defesa \textbf{híbrida}. A prevenção tenta eliminar a possibilidade de ataques de DDoS, isto é, evita a negação de serviço para os clientes legítimos. A linha de reação detecta o ataque e responde imediatamente, e a abordagem híbrida combina os métodos anteriores, não só prevenindo mas também reagindo à ataques. 
%A maioria das soluções de segurança são construídas focadas na prevenção de ataques \cite{4429182}. 
%As abordagens clássicas buscam reduzir os riscos a zero, o que é impraticável para todos os tipos de riscos, pois podem ser custosos e complexos.
%Contudo, os atuais sistemas em redes são tão complexos que é impossível identificar e corrigir todas as suas vulnerabilidades antes que elas se tornem ataques ou intrusões e, muitas vezes, impossível de se recuperar de falhas decorrentes. Logo, a garantia de funcionamento destes sistemas sob condições de ataques ou intrusões têm se tornado uma prioridade. Diversos autores, como \cite{Verissimo} e \cite{4796927}, têm defendido que as linhas clássicas de defesa não apresentam abordagens que garantam a tolerância a falhas em caso de ataques ou intrusões. Assim, a abordagem de \textbf{tolerância a intrusão}, a qual lida com sistemas de segurança que garantem o funcionamento dos sistemas mesmo que estejam sob problemas de segurança, minimizando ao máximo os prejuízos até o retorno ao fluxo normal de funcionamento \cite{Fraga_Powell_1985}. Ao invés de tentar se prevenir de todo tipo de ataque ou intrusão, os sistemas passam a tolerar os problemas simples de segurança e criam mecanismos de controle à intrusão  de forma que eles não causem nenhuma falha ao sistema. Esta abordagem faz uso de replicação ou redundância para garantir os aspectos de tolerância e são aplicados,especificamente, quando a forma de intrusão ou ataque seja diferenciada ou desconhecida \cite{4796927}. 

Este trabalho propõe uma arquitetura reativa e tolerante a falhas para a mitigação de ataques de DDoS executados contra aplicações hospedadas em uma \emph{cloud}. Tal arquitetura é baseada na instanciação de uma réplica da aplicação e no redirecionamento apenas de requisições legítimas a esta réplica.  A arquitetura monitora o tráfego de uma aplicação e ao detectar uma possível anomalia, isto é, a ocorrência de um ataque de DDoS, ela estabelece uma nova instância desta aplicação, garantindo que nenhum tráfego malicioso a alcance. 
As diferenças desta solução para outras propostas são que a aplicação hospedada não precisa prover acurácia na filtragem de tráfego legítimo, o uso dos recursos não é onerado financeiramente, e a intervenção humana é desnecessária. 
%A solução é inovadora porque não precisa identificar os clientes atacantes e, ainda assim, consegue filtrar apenas o tráfego legítimo sem a carga e possíveis erros de categorização que seriam introduzidos pela tentativa de identificação de clientes. 
Uma avaliação experimental considerando o tempo de resposta aos clientes %, bem como a sobrecarga ao sistema, 
mostra a eficácia de uma implementação da arquitetura diante de ataques DDoS a um serviço Web.
 
%Embora os ataques de DDoS ainda persistam e tenham crescido ultimamente, dadas as capacidades das redes, este tipo de ataque pode ser considerado comum e bem conhecido pelos sistemas de segurança existentes executando em arquiteturas comuns. Por este motivo, o seu tratamento pode ser incluído dentro da abordagem clássica. Assim, este trabalho de pesquisa propõe uma arquititetura de mitigação classificada como reativa, pois verifica o tráfego e reage as suas anomalias, e tolerante a falhas por garantir o funcionamento da aplicação mesmo sob ataque, através de replicação da aplicação original que passa a servir apenas como redirecionador de tráfego.

O restante do artigo está organizado da seguinte maneira: a Seção 2 apresenta os trabalhos relacionados. A Seção 3 detalha a arquitetura proposta para a mitigação de ataques \emph{DDoS}. A Seção 4 apresenta uma descrição da implementação realizada da arquitetura. A Seção 5 apresenta uma avaliação, juntamente com o cenário e os resultados. Por fim, a conclusão e trabalhos futuros são apresentados na Seção 6.
