%!TEX root = ../proposta.tex

Many researches and proposes have been developed looking for solutions for current problems of the Internet, which are propagating to the Internet of the Future (IoF). Such problems may be largely categorized in the fields of mobility, quality of service and security, which are heading to acceptable solutions, aggravated by the appearance of new architectures. Currently we have data and applications hosted on distinct and unknown physical locations. Another great change that has occurred in the way of managing a system, what earlier was on a local scope, with characteristics users and servers. Now, such system is hosted on environments built by shared resources of many heterogeneous Autonomous Systems (AS) \cite{5486552}.

The massive use of resources available on the Internet and all the connectivity provided by this computational environment with services for personal, commercial or academic use, makes this environment a target aimed by malicious softwares. Furthermore, this is aggravated by the way the TCP/IP architecture may favor an attacker. The Internet Protocol (IP) omits information about a sender's true identity, allowing them to remain anonymous and unpunished. \cite{1039856}

Despite many years of researchers' efforts, one type of attack still poses serious threats to many Internet servers: the Denial of Service (DoS) attack. It's one of the major security challenges currently propagated to the IoF, which interconnects many more devices and people.

One way of defense against this kind of attack is prevention and reaction. The DoS attack does not intend to invade any computer to acquire classified information, nor change information stored on it. Its goal is to make unavailable a service that's being provided, sending a massive traffic of requests to the service host. This way, the service cannot answer to all requests sent to it.

The problem becomes even more severe when many traffic generators intensify the traffic being sent in a distributed way, featuring a Distributed Denial of Service (DDoS) attack \cite{Sachdeva08ddosincidents}. The aftermath is the freeze, restart, or even complete depletion of host's resources. The most affected services are those which allow anonymous requests, such as web services. Thus, the challenge in eliminating DDoS attacks relies on the difficulty of tell the difference between authentic requests and requests from attackers \cite{Li:2009:DDA:1683304.1684620}.

With the new network and application architectures that configures the IoF, there are complex and robust systems such as Clouds, where the challenge of mitigating this kind of attack becomes even more necessary. Great part of the usually offered solutions to mitigate DDoS attacks on Clouds are based entirely in allocating more resources \cite{Peng:2007:SND:1216370.1216373}, but the costs to maintain such resources will also be increased. This behavior is known as Economic DDoS (eDDoS \cite{Soon:10}).

Some unusual approaches are unfit for taking premises that are not always true or for being too costly \cite{Bakshi:10}, \cite{Liu:2010:NFD:1866835.1866849}. There are many records of attacks that have shaken the Internet lately, such as those against Yahoo!, eBay, Amazon.com and many others popular sites in February of 2000. In the beginning of 2011 happened the attack against the blogs host WordPress, that faced the worst DDoS attack on its history \cite{infoexame}.

These kind of attacks are triggered in alarming proportions, according to recent discoveries that also reveal the techniques applied to create zombies networks. One example is the network created by the TDL-4 worm, which is classified by security experts as "not perfectly, but pratically undesctructable", with nearly 4.5 million nodes only in 2011 \cite{tdl4}. Since February of 2010, the hacker activist group known as Anonymous started a series of attacks of political and ideological nature against many international institutions \cite{titstorm}. The most massive part of those attacks were made of DDoS's \cite{infoexame}. Thus, DDoS mitigation in Clouds still demand more researches.

The goal of this work is elaborate an architecture for mitigating DDoS attacks against an application hosted on a Cloud. This architecture should monitor the application traffic, and, when a possible DDoS is detected, create a new instance of the application, in a way to guarantee that no malicious traffic will be able to reach it.

The need to protect network architectures, and mitigate DDoS attacks has been recognized both in the academic and commercial environment. According to \cite{1039856}, there are 3 main defensive lines against DDoS attacks, which are \textbf{prevention}, \textbf{reaction}, or both, in a defense called \textbf{hybrid}. Prevention tries to eliminate the conditions to DDoS attack, I.E., avoids the Denial of Service to authentic clients. The Reaction detects the attack and responds immediately, while the Hybrid approach combines those two methods, not only preventing, but also reacting to attacks. Most part of security systems are built focusing on attack prevention \cite{4429182}. The classic approaches focus on reducing risks near to zero, what is unworkable to every type of risk, since it may be very costly and complex.

Currently, systems have become so complex that it's impossible to identify and correct all vulnerabilities before they turn into attacks or invasions, and, many times, impossible to recover from due failures. Besides, the guarantee of functioning of those systems under attacks or intrusions conditions have become a priority. Many authors like \cite{Verissimo},  \cite{4796927} and \cite{1424871} have defended that the classic defensive lines do not guarantee fault-tolerance in case of attack or intrusion.

Thus, a new approach was defined, called \textbf{intrusion-tolerant}, which deals with security systems that assure the functioning of systems even under security problems, reducing damages until the traffic returns to normal levels \cite{Fraga_Powell_1985}. Instead trying to prevent avery type of attack or intrusion, systems come to tolerate simple security problems and create intrusion control mechanisms in a way that it won't cause any system failure. This approach uses replication and redundancy to guarantee tolerance, and are applied, specifically, when the intrusion or attack technique is unknown.
  
Although DDoS attacks still persist, and have become bigger lately, given the network capacity, this king of attack may be considered common and well-known by the existing security systems executing in common architectures. Due to that, its treatment may be included in a classic approach. Thus, this research work proposes a mitigation architecture classified as reactive, since it verifies the traffic and reacts to annomalies, and fault-tolerant, for it guarantees the functioning of an application even under attack, through replication of the original application, that starts acting just as a traffic redirector.

This work is divided according to the following description. For starters, the first section presents and introduction to the subject. Following up, section 2 exposes the related works, followed by the proposed architecture, on section 3. Section 4 presents a description of the implementation. Then, an evaluation is in section 5, with the scenario and results. At last, conclusion and future works are in section 6.