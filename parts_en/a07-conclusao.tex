%!TEX root = ./proposta3.tex

Este trabalho apresentou uma proposta de arquitetura de mitigação para ataques de DDoS direcionados à aplicações web hospedadas em \emph{cloud}. A arquitetura é dependente apenas da existência do ambiente na \emph{cloud}. Os quatro sub módulos especificados garantem o funcionamento das etapas de forma autônoma até mesmo para gerar uma nova instância da aplicação. Assim como, permite livre acesso para os clientes legítimos, dado que apenas eles realizarão o redirecionamento solicitado. O uso da \emph{blacklist} tem gerencimento eficiente pelo uso de tempo de validade, que no caso de atacantes, terá aumento exponencial para reincidências.

Os resultados alcançados nas experimentações demonstram a validade da proposta de arquitetura de mitigação a ataques de DDoS pois, não onera financeiramente o fornecedor da aplicação, apresenta baixo \emph{overhead} ao ser incluído no ambiente da aplicação e consegue direcionar o tráfego legítimo de modo satisfatório, impossibilitando os atacantes de acessarem a nova instância criada.
Outra caractística considerável da arquitetura é a tolerância a falhas mesmo em cenários de ataques, assim como a o baixo tempo de resposta as requisções e nenhuma perda na taxa de entrega.   

A arquitetura proposta trabalha a filtragem e redirecionamento de requisições legítimas para uma nova aplicação sem incidência de ataques, porém ela considera que a aplicação original está sobre ataque pela quantidade de tráfego que recebe. Mecanismos mais robustos para a detecção de DDoS podem ser desenvolvidos como trabalhos futuros complementando a solução já implementada, garantindo que esta apresente uma alta acurácia na detecção de tipos específicos de ataque de DDoS. Outra proposta de trabalhos futuros diz respeito a implementação e checagem da \emph{blacklist} em níveis mais baixos e otimizados como em servidores HTTP.